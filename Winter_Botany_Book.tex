\documentclass[openany]{book}
\usepackage{lmodern}
\usepackage{amssymb,amsmath}
\usepackage{ifxetex,ifluatex}
\usepackage{fixltx2e} % provides \textsubscript
\ifnum 0\ifxetex 1\fi\ifluatex 1\fi=0 % if pdftex
  \usepackage[T1]{fontenc}
  \usepackage[utf8]{inputenc}
\else % if luatex or xelatex
  \ifxetex
    \usepackage{mathspec}
  \else
    \usepackage{fontspec}
  \fi
  \defaultfontfeatures{Ligatures=TeX,Scale=MatchLowercase}
    \setmainfont[]{Times New Roman}
\fi
% use upquote if available, for straight quotes in verbatim environments
\IfFileExists{upquote.sty}{\usepackage{upquote}}{}
% use microtype if available
\IfFileExists{microtype.sty}{%
\usepackage{microtype}
\UseMicrotypeSet[protrusion]{basicmath} % disable protrusion for tt fonts
}{}
\usepackage{hyperref}
\hypersetup{unicode=true,
            pdftitle={Winter Botany in the British Isles},
            pdfauthor={Max Brown},
            pdfborder={0 0 0},
            breaklinks=true}
\urlstyle{same}  % don't use monospace font for urls
\usepackage{natbib}
\bibliographystyle{apalike}
\usepackage{graphicx,grffile}
\makeatletter
\def\maxwidth{\ifdim\Gin@nat@width>\linewidth\linewidth\else\Gin@nat@width\fi}
\def\maxheight{\ifdim\Gin@nat@height>\textheight\textheight\else\Gin@nat@height\fi}
\makeatother
% Scale images if necessary, so that they will not overflow the page
% margins by default, and it is still possible to overwrite the defaults
% using explicit options in \includegraphics[width, height, ...]{}
\setkeys{Gin}{width=\maxwidth,height=\maxheight,keepaspectratio}
\IfFileExists{parskip.sty}{%
\usepackage{parskip}
}{% else
\setlength{\parindent}{0pt}
\setlength{\parskip}{6pt plus 2pt minus 1pt}
}
\setlength{\emergencystretch}{3em}  % prevent overfull lines
\providecommand{\tightlist}{%
  \setlength{\itemsep}{0pt}\setlength{\parskip}{0pt}}
\setcounter{secnumdepth}{0}
% Redefines (sub)paragraphs to behave more like sections
\ifx\paragraph\undefined\else
\let\oldparagraph\paragraph
\renewcommand{\paragraph}[1]{\oldparagraph{#1}\mbox{}}
\fi
\ifx\subparagraph\undefined\else
\let\oldsubparagraph\subparagraph
\renewcommand{\subparagraph}[1]{\oldsubparagraph{#1}\mbox{}}
\fi

%%% Use protect on footnotes to avoid problems with footnotes in titles
\let\rmarkdownfootnote\footnote%
\def\footnote{\protect\rmarkdownfootnote}

%%% Change title format to be more compact
\usepackage{titling}

% Create subtitle command for use in maketitle
\providecommand{\subtitle}[1]{
  \posttitle{
    \begin{center}\large#1\end{center}
    }
}

\setlength{\droptitle}{-2em}

  \title{Winter Botany in the British Isles}
    \pretitle{\vspace{\droptitle}\centering\huge}
  \posttitle{\par}
    \author{Max Brown}
    \preauthor{\centering\large\emph}
  \postauthor{\par}
      \predate{\centering\large\emph}
  \postdate{\par}
    \date{2019-11-03}

\usepackage{booktabs}
\usepackage{amsthm}
\makeatletter
\def\thm@space@setup{%
  \thm@preskip=8pt plus 2pt minus 4pt
  \thm@postskip=\thm@preskip
}
\makeatother

\begin{document}
\maketitle

{
\setcounter{tocdepth}{1}
\tableofcontents
}
\hypertarget{preface}{%
\chapter*{Preface}\label{preface}}
\addcontentsline{toc}{chapter}{Preface}

I am a strong believer in a top down approach when it comes to
identification. In the context here, this means trying to
compartmentalise hundreds of species into natural groups such as orders,
families or genera. Not only does it allow you to not have to worry
about species (for the moment), but it allows you to see evolutionary
groups of plants more easily. When confronted with new species in other
parts of the world, with luck they will be placed within your working
knowlegde.

This guide is not intended as a replacement to the guides that already
exist but it can be viewed to supplement them. By its nature of only
dealing with genera the keys are necessarily shorter and less complex,
which is the idea of this guide. Where possible the terminology is
minimised and only the major characters of use are mentioned. This key
will have very little in it that is actually new, at least in its
current form; it will be a synthesis of what in the literature already
exists. So for this I cannot take credit. I have relied heavily on the
literature, in no particular order: Winter Botany by William Trelease
(date?), The Vegetative Key to the British Flora by John Poland and Eric
Clement (date?), Flora of the British Isles by Clive Stace (2012), Trees
of Britain and Northern Europe by Alan Mitchell (1976), more recently
Identification of trees and shrubs in winter using buds and twigs by
Bernd Schulz (2018) and the Conifers Plant Crib at the BSBI (1998) by
Peter Sell,
\url{http://bsbi.org/wp-content/uploads/dlm_uploads/Conifers_Crib_3.pdf}.
As always, where I could, terminology was simplified, though a glossary
will be provided at the end.

The key to broad sections is outlined first as this is a winter
botanical key, not all species will be in leaf. First thing to note is
that I have separated out the conifers first, because I believe them to
have such a distinct structure compared to flowering plants that it
would be confusing to integrate them with the other keys. This is not
the case for deciduous coniferous species, which have been included in
the main keys. When there are no leaves on a coniferous plant, it is
difficult to tell that a deciduous conifer is not a flowering plant.
Luckily, there are only four genera that fit into this category
\emph{Larix}, \emph{Gingko}, \emph{Metasequoia} and \emph{Taxodium}.

The key should cater for any woody plant you find in the winter in the
UK, save a few rare hortal escapes (though many of these are also
covered), whether you are at the top of a mountain or by the sea.

My interest was sparked in the winter of 2016, mostly because I was
frustrated I knew nothing about twigs! I have been compiling this on and
off since then.

\hypertarget{introduction}{%
\chapter*{Introduction}\label{introduction}}
\addcontentsline{toc}{chapter}{Introduction}

Many woody plants in (especially northern) temperate climates with
seasonal weather shed their leaves in the colder months of the year. On
the most recent years growth, leaves fall to reveal the twig beneath. It
is these structures left behind after the leaves fall that are valuable
in identification. The twig itself can vary in thickness, colour, growth
pattern (straight or zig-zagged) and cross sectional shape of the twig
(round, oval, square).

Where the leaf falls a scar is left behind. The distribution of leaf
scars on a twig (phyllotaxy) is of major importance. The presence of at
least one leaf scar indicates where the node is on a twig. Leaf scars
can be present in groups of three or more, in which case they are
considered whorled. Secondly leaf scars can be paired, usually on
opposite sides of the twig. Thirdly and most commonly, leaf scars are
present singularly at a node.Leaf scars are characteristic in size,
shape and the number and distribution of vascular bundles present within
them.

Above the leaf scar, a bud will be present which contains meristems that
will continue next years growth. They are often surrounded by scales,
the size, shape and number of which are often diagnostic. Buds can be
multiple at one node either next to each other (collateral), or above
one another (superposed). The presence or absence of an \emph{end} or
\emph{terminal} bud is determined either by the absence of an obvious
leaf scar or by its morphological difference to lateral buds. Terminal
buds are usually larger, if present. Sometimes buds can appear to be
different even on the lateral buds and this is usually associated with a
difference between buds that will go on to produce flowers, and those
that will continue vegetative growth.

Twigs can have many and varied outgrowths on them. These vary from
small, thin flexible outgrowths (hairs), which can be shaped simply,
forked, tree-like, woolly, star-shaped and glandular to name a few.
Hardened outgrowths of the stem are called \emph{prickles}. In contrast,
\emph{spines} are modified leaves and therefore have a bud at the base
of them (be careful, bud may be obscured!) and \emph{thorns} are
modified stems.

Where there is only one species in the genus, or the key specifically
arrives at a species, the species is indicated in parentheses.

\hypertarget{general-key}{%
\chapter*{General Key}\label{general-key}}
\addcontentsline{toc}{chapter}{General Key}

\begin{enumerate}
\def\labelenumi{\arabic{enumi}.}
\item
  Leaves evergreen, needle like, scale like or both, usually hard,
  mostly resinous or odorous. Reproducing through cones -\textgreater{}
  Key 1 (Conifers)
\item
  Leaves evergreen, deciduous or absent, not needle or scale like,
  resinous or not. Reproducing (almost exclusively) through flowers and
  fruits -\textgreater{} 3 (Flowering plants)

  \begin{enumerate}
  \def\labelenumii{\arabic{enumii}.}
  \setcounter{enumii}{2}
  \tightlist
  \item
    Leaves evergreen and parallel veined -\textgreater{} Key 2
  \item
    If leaves evergreen, then not parallel veined -\textgreater{} 5
  \end{enumerate}
\item
  Leaves or leaf scars \textgreater{}2 per node (whorled, pseudowhorled
  or in bundles) -\textgreater{} Key 3
\item
  Leaves or leaf scars \textless{}3 per node -\textgreater{} 7

  \begin{enumerate}
  \def\labelenumii{\arabic{enumii}.}
  \setcounter{enumii}{6}
  \tightlist
  \item
    Leaves or leaf scars 2 per node (opposite or subopposite)
    -\textgreater{} Key 4
  \item
    Leaves or leaf scars 1 per node (alternate) -\textgreater{} Key 5
  \end{enumerate}
\end{enumerate}

\hypertarget{conifers}{%
\chapter*{Conifers}\label{conifers}}
\addcontentsline{toc}{chapter}{Conifers}

\hypertarget{key-1-coniferous-plants-reproducing-by-cones}{%
\section*{Key 1: Coniferous plants reproducing by
cones}\label{key-1-coniferous-plants-reproducing-by-cones}}
\addcontentsline{toc}{section}{Key 1: Coniferous plants reproducing by
cones}

Conifers are seed plants without flowers, instead reproducing through
cones. Only a few species are deciduous and the rest are evergreen. Most
coniferous genera are highly distinct, with discrete and interesting
characters.

\begin{enumerate}
\def\labelenumi{\arabic{enumi}.}
\tightlist
\item
  Leaves whorled. Either whorled on short shoots and needle like,
  3(4)-whorled around a twig or in large distant whorls -\textgreater{}
  Key 1.1
\item
  Leaves not whorled -\textgreater{} 3

  \begin{enumerate}
  \def\labelenumii{\arabic{enumii}.}
  \setcounter{enumii}{2}
  \tightlist
  \item
    Leaves needle like, in bundles of 2's, 3's or 5's -\textgreater{}
    \emph{Pinus}
  \item
    If leaves needle like, not bundled -\textgreater{} 5
  \end{enumerate}
\item
  Leaves strictly opposite, adnate, scale like -\textgreater{} Key 1.2
\item
  Leaves alternate or spiral. Adnate, scale like or not -\textgreater{}
  7

  \begin{enumerate}
  \def\labelenumii{\arabic{enumii}.}
  \setcounter{enumii}{6}
  \tightlist
  \item
    Young twigs brown or grey (by year 2). If shoot ribbed, not green
    -\textgreater{} Key 1.3
  \item
    Young twigs green (until year 3). Shoot ribbed by green leaf bases
    -\textgreater{} Key 1.4
  \end{enumerate}
\end{enumerate}

\hypertarget{key-1.1-leaves-whorled}{%
\subsection*{Key 1.1: Leaves whorled}\label{key-1.1-leaves-whorled}}
\addcontentsline{toc}{subsection}{Key 1.1: Leaves whorled}

\begin{enumerate}
\def\labelenumi{\arabic{enumi}.}
\tightlist
\item
  Leaves deciduous, cones persistent\ldots{} \emph{Larix}
\item
  Leaves evergreen -\textgreater{} 3

  \begin{enumerate}
  \def\labelenumii{\arabic{enumii}.}
  \setcounter{enumii}{2}
  \tightlist
  \item
    Leaves in rosettes of 20-80 leaves on short shoots \ldots{}
    \emph{Cedrus}
  \item
    Leaves never on short shoots -\textgreater{} 5
  \end{enumerate}
\item
  Leaves many, \textgreater{}7, 70-120mm long in large, distant whorls
  \ldots{} \emph{Sciadopitys}
\item
  Leaves in 3's, rarely 4's -\textgreater{} 7

  \begin{enumerate}
  \def\labelenumii{\arabic{enumii}.}
  \setcounter{enumii}{6}
  \tightlist
  \item
    Leaves blunt tipped \ldots{} \emph{Fitzroya} (cupressoides)
  \item
    Leaves sharply pointed \ldots{} \emph{Juniperus}
  \end{enumerate}
\end{enumerate}

\hypertarget{key-1.2-leaves-strictly-opposite-scale-like-cupressaceae}{%
\subsection*{Key 1.2: Leaves strictly opposite, scale like
(Cupressaceae)}\label{key-1.2-leaves-strictly-opposite-scale-like-cupressaceae}}
\addcontentsline{toc}{subsection}{Key 1.2: Leaves strictly opposite,
scale like (Cupressaceae)}

\begin{enumerate}
\def\labelenumi{\arabic{enumi}.}
\tightlist
\item
  Leaves broader, with large white patches below\ldots{}
  \emph{Thujopsis} (dolabrata)
\item
  Leaves smaller, not broad -\textgreater{} 3

  \begin{enumerate}
  \def\labelenumii{\arabic{enumii}.}
  \setcounter{enumii}{2}
  \tightlist
  \item
    Shoots with at least some leaves in whorls of 3 (check young
    foliage)\ldots{} \emph{Juniperus}
  \item
    Leaves never 3-whorled -\textgreater{} 5
  \end{enumerate}
\item
  Branchlets spreading in 3 dimensions. Twigs rounded or 4-sided. Scale
  leaves all similar. Female cone thickly woody\ldots{} \emph{Cupressus}
\item
  Branchlets arranged in 1 plane, or rarely 3 dimensions. Twigs flat, or
  slightly flat and 4 sided. -\textgreater{} 7

  \begin{enumerate}
  \def\labelenumii{\arabic{enumii}.}
  \setcounter{enumii}{6}
  \tightlist
  \item
    Young shoots slightly flat and nearly 4 sided. One common cultivar
    with branchlets in 3 dimensions\ldots{} \emph{x Cupressocyparis}
    (leylandii)
  \item
    Young shoots very flat, lateral scale leaves keeled -\textgreater{}
    9
  \end{enumerate}
\item
  Scale leaves same colour on both sides, scentless foliage when
  crushed\ldots{} \emph{Platycladus} (orientalis)
\item
  Scale leaves lighter or whitish at margins below -\textgreater{} 11

  \begin{enumerate}
  \def\labelenumii{\arabic{enumii}.}
  \setcounter{enumii}{10}
  \tightlist
  \item
    Female cones globular and woody, terminal shoots whip-like\ldots{}
    \emph{Chamaecyparis}
  \item
    Female cones flask shaped with scales overlapping, terminal shoots
    erect\ldots{} \emph{Thuja} (plicata)
  \end{enumerate}
\end{enumerate}

\hypertarget{key-1.3-young-twigs-brown-or-grey}{%
\subsection*{Key 1.3: Young twigs brown or
grey}\label{key-1.3-young-twigs-brown-or-grey}}
\addcontentsline{toc}{subsection}{Key 1.3: Young twigs brown or grey}

\begin{enumerate}
\def\labelenumi{\arabic{enumi}.}
\tightlist
\item
  Leaves with sucker like base, attached directly to twig\ldots{}
  \emph{Abies}
\item
  Leaves without sucker like base, attached to a projection on twig
  -\textgreater{} 3

  \begin{enumerate}
  \def\labelenumii{\arabic{enumii}.}
  \setcounter{enumii}{2}
  \tightlist
  \item
    Leaves with indistinct petiole, twig very rough when leaves
    fallen\ldots{} \emph{Picea}
  \item
    Leaves with obvious petiole, bare shoots slightly rough
    -\textgreater{} 5
  \end{enumerate}
\item
  Leaves minutely serrate, buds hidden by leaves\ldots{} \emph{Tsuga}
\item
  Leaves entire, buds prominent\ldots{} \emph{Pseudotsuga} (menziesii)
\end{enumerate}

\hypertarget{key-1.4-young-twigs-green}{%
\subsection*{Key 1.4: Young twigs
green}\label{key-1.4-young-twigs-green}}
\addcontentsline{toc}{subsection}{Key 1.4: Young twigs green}

\begin{enumerate}
\def\labelenumi{\arabic{enumi}.}
\tightlist
\item
  Leaves with a petiole\ldots{} \emph{Taxus} (baccata)
\item
  Leaves sessile -\textgreater{} 3

  \begin{enumerate}
  \def\labelenumii{\arabic{enumii}.}
  \setcounter{enumii}{2}
  \tightlist
  \item
    Leaves of two kinds, scale like, and linear/flat\ldots{}
    \emph{Sequoia} (sempervirens)
  \item
    Leaves all similar -\textgreater{} 5
  \end{enumerate}
\item
  Leaves \textgreater{}25mm long, \textgreater{}10mm wide, sharply spine
  tipped\ldots{} \emph{Araucaria} (araucana)
\item
  Leaves \textless{}25mm long -\textgreater{} 7

  \begin{enumerate}
  \def\labelenumii{\arabic{enumii}.}
  \setcounter{enumii}{6}
  \tightlist
  \item
    Leaves spirally arranged, with free part 3-7mm\ldots{}
    \emph{Sequoiadendron} (giganteum)
  \item
    Leaves in 5 ranks, incurved, with free part 5-20mm\ldots{}
    \emph{Cryptomeria} (japonica)
  \end{enumerate}
\end{enumerate}

Introduction

I am a strong believer in a top down approach when it comes to
identification. The top down approach in this context means trying to
compartmentalise hundreds of species into natural groups such as orders,
families or genera. Not only does it allow you to not have to worry
about species (for the moment), but it allows you to see evolutionary
groups of plants more easily. When confronted with new species in other
parts of the world, with luck they will be placed within your working
framework.

This guide is not intended as a replacement to the guides that already
exist but it can be viewed to supplement them. By its nature of only
dealing with genera the keys are necessarily shorter and less complex,
which is the idea of this guide. Where possible the terminology is
minimised and only the major characters of use are mentioned. This key
will have very little in it that is actually new, it will be a synthesis
of what in the literature already exists. So for this I cannot take
credit. I have relied heavily on the literature, in no particular order:
Winter Botany by William Trelease (date?), The Vegetative Key to the
British Flora by John Poland and Eric Clement (date?), Flora of the
British Isles by Clive Stace (2012), Trees of Britain and Northern
Europe by Alan Mitchell (1976), more recently Identification of trees
and shrubs in winter using buds and twigs by Bernd Schulz (2018) and the
Conifers Plant Crib at the BSBI (1998) by Peter Sell,
\url{http://bsbi.org/wp-content/uploads/dlm_uploads/Conifers_Crib_3.pdf}.
As always, where I could, terminology was simplified, though a glossary
will be provided at the end.

The key to broad sections is outlined below. As this is a winter
botanical key, not all species will be in leaf. First thing to note is
that I have separated out the conifers first, because I believe them to
have such a distinct structure compared to flowering plants that it
would be confusing to integrate them with the other keys. This is not
the case for deciduous coniferous species, which have been included in
the main keys. This is because when there are no leaves, it is difficult
to tell that a deciduous conifer is not a flowering plant. These genera
are \emph{Larix}, \emph{Gingko}, \emph{Metasequoia} and \emph{Taxodium}.

Prickles are acute outgrowth of the stem, spines are modified leaves and
thorns are modified stems.

The key should cater for any woody plant you find in the winter in the
UK, save a few rare hortal escapes (though many of these are also
covered), whether you are at the top of a mountain or by the sea.

My interest was sparked in the winter of 2016, mostly because I was
frustrated I knew nothing about twigs! I have been compiling this on and
off since then.

\hypertarget{keys}{%
\chapter{Keys}\label{keys}}

\hypertarget{general-key-1}{%
\section{General Key}\label{general-key-1}}

\begin{enumerate}
\def\labelenumi{\arabic{enumi}.}
\item
  Leaves evergreen, needle like, scale like or both, usually hard,
  mostly resinous or odorous. Reproducing through cones -\textgreater{}
  Key 1 (Conifers)
\item
  Leaves evergreen, deciduous or absent, not needle or scale like,
  resinous or not. Reproducing (almost exclusively) through flowers and
  fruits -\textgreater{} 3 (Flowering plants)

  \begin{enumerate}
  \def\labelenumii{\arabic{enumii}.}
  \setcounter{enumii}{2}
  \tightlist
  \item
    Leaves evergreen and parallel veined -\textgreater{} Key 2
  \item
    If leaves evergreen, then not parallel veined -\textgreater{} 5
  \end{enumerate}
\item
  Leaves or leaf scars \textgreater{}2 per node (whorled, pseudowhorled
  or in bundles) -\textgreater{} Key 3
\item
  Leaves or leaf scars \textless{}3 per node -\textgreater{} 7

  \begin{enumerate}
  \def\labelenumii{\arabic{enumii}.}
  \setcounter{enumii}{6}
  \tightlist
  \item
    Leaves or leaf scars 2 per node (opposite or subopposite)
    -\textgreater{} Key 4
  \item
    Leaves or leaf scars 1 per node (alternate) -\textgreater{} Key 5
  \end{enumerate}
\end{enumerate}

\hypertarget{key-1-coniferous-plants-reproducing-by-cones-1}{%
\section{Key 1: Coniferous plants reproducing by
cones}\label{key-1-coniferous-plants-reproducing-by-cones-1}}

Conifers are seed plants without flowers, instead reproducing through
cones. Only a few species are deciduous and the rest are evergreen. Most
coniferous genera are highly distinct, with discrete and interesting
characters.

\begin{enumerate}
\def\labelenumi{\arabic{enumi}.}
\tightlist
\item
  Leaves whorled. Either whorled on short shoots and needle like,
  3(4)-whorled around a twig or in large distant whorls -\textgreater{}
  Key 1.1
\item
  Leaves not whorled -\textgreater{} 3

  \begin{enumerate}
  \def\labelenumii{\arabic{enumii}.}
  \setcounter{enumii}{2}
  \tightlist
  \item
    Leaves needle like, in bundles of 2's, 3's or 5's -\textgreater{}
    \emph{Pinus}
  \item
    If leaves needle like, not bundled -\textgreater{} 5
  \end{enumerate}
\item
  Leaves strictly opposite, adnate, scale like -\textgreater{} Key 1.2
\item
  Leaves alternate or spiral. Adnate, scale like or not -\textgreater{}
  7

  \begin{enumerate}
  \def\labelenumii{\arabic{enumii}.}
  \setcounter{enumii}{6}
  \tightlist
  \item
    Young twigs brown or grey (by year 2). If shoot ribbed, not green
    -\textgreater{} Key 1.3
  \item
    Young twigs green (until year 3). Shoot ribbed by green leaf bases
    -\textgreater{} Key 1.4
  \end{enumerate}
\end{enumerate}

\hypertarget{key-1.1-leaves-whorled-1}{%
\subsection{Key 1.1: Leaves whorled}\label{key-1.1-leaves-whorled-1}}

\begin{enumerate}
\def\labelenumi{\arabic{enumi}.}
\tightlist
\item
  Leaves deciduous, cones persistent\ldots{} \emph{Larix}
\item
  Leaves evergreen -\textgreater{} 3

  \begin{enumerate}
  \def\labelenumii{\arabic{enumii}.}
  \setcounter{enumii}{2}
  \tightlist
  \item
    Leaves in rosettes of 20-80 leaves on short shoots \ldots{}
    \emph{Cedrus}
  \item
    Leaves never on short shoots -\textgreater{} 5
  \end{enumerate}
\item
  Leaves many, \textgreater{}7, 70-120mm long in large, distant whorls
  \ldots{} \emph{Sciadopitys}
\item
  Leaves in 3's, rarely 4's -\textgreater{} 7

  \begin{enumerate}
  \def\labelenumii{\arabic{enumii}.}
  \setcounter{enumii}{6}
  \tightlist
  \item
    Leaves blunt tipped \ldots{} \emph{Fitzroya} (cupressoides)
  \item
    Leaves sharply pointed \ldots{} \emph{Juniperus}
  \end{enumerate}
\end{enumerate}

\hypertarget{key-1.2-leaves-strictly-opposite-scale-like-cupressaceae-1}{%
\subsection{Key 1.2: Leaves strictly opposite, scale like
(Cupressaceae)}\label{key-1.2-leaves-strictly-opposite-scale-like-cupressaceae-1}}

Cupressus - foliage in spray like plumes, branchlets at varying angles.
Twigs rounded or 4-sided, the scale-leaves even on all sides

x Cupressocyparis - foliage in flattened pinnate sprays, cone globular
and woody. Young shoots only slightly flat, nearly 4-sided

Chamaecyparis - foliage in flattened pinnate sprays, cone globular and
woody. Twigs flat, the facial scale-leaves usually flat, rarely keeled,
the lateral scale-leaves keeled. Young shoots distinctly flat. Terminal
shoot usually `whip-like', drooping; cones globose, the peltate scales
touching only at the margins

Thuja - foliage in flattened pinnate sprays, cone flask shaped,
leathery. Twigs flat, the facial scale-leaves usually flat, rarely
keeled, the lateral scale-leaves keeled. Young shoots distinctly flat.
Terminal shoot erect; cones ovate to oblong, their scales overlapping.
Foliage spreading in flat sprays, aromatic when crushed; scale-leaves a
different colour on lower sides from upper

Platycladus -foliage in flattened pinnate sprays, cone flask shaped,
leathery. Foliage scentless. Twigs flat, the facial scale-leaves usually
flat, rarely keeled, the lateral scale-leaves keeled. Young shoots
distinctly flat. Foliage in vertical sprays, without scent when crushed;
scale-leaves the same colour on both sides

\begin{enumerate}
\def\labelenumi{\arabic{enumi}.}
\tightlist
\item
  Leaves broader, with large white patches below\ldots{}
  \emph{Thujopsis} (dolabrata)
\item
  Leaves smaller, not broad -\textgreater{} 3

  \begin{enumerate}
  \def\labelenumii{\arabic{enumii}.}
  \setcounter{enumii}{2}
  \tightlist
  \item
    Shoots with at least some leaves in whorls of 3 (check young
    foliage)\ldots{} \emph{Juniperus}
  \item
    Leaves never 3-whorled -\textgreater{} 5
  \end{enumerate}
\item
  Branchlets spreading in 3 dimensions. Twigs rounded or 4-sided. Scale
  leaves all similar. Female cone thickly woody\ldots{} \emph{Cupressus}
\item
  Branchlets arranged in 1 plane, or rarely 3 dimensions. Twigs flat, or
  slightly flat and 4 sided. -\textgreater{} 7

  \begin{enumerate}
  \def\labelenumii{\arabic{enumii}.}
  \setcounter{enumii}{6}
  \tightlist
  \item
    Young shoots slightly flat and nearly 4 sided. One common cultivar
    with branchlets in 3 dimensions\ldots{} \emph{x Cupressocyparis}
    (leylandii)
  \item
    Young shoots very flat, lateral scale leaves keeled -\textgreater{}
    9
  \end{enumerate}
\item
  Scale leaves same colour on both sides, scentless foliage when
  crushed\ldots{} \emph{Platycladus} (orientalis)
\item
  Scale leaves lighter or whitish at margins below -\textgreater{} 11

  \begin{enumerate}
  \def\labelenumii{\arabic{enumii}.}
  \setcounter{enumii}{10}
  \tightlist
  \item
    Female cones globular and woody, terminal shoots whip-like\ldots{}
    \emph{Chamaecyparis}
  \item
    Female cones flask shaped with scales overlapping, terminal shoots
    erect\ldots{} \emph{Thuja} (plicata)
  \end{enumerate}
\end{enumerate}

Olfactory table:

Thujopsis Juniperus - smells like gin? Cupressus x Cupressocyparis
Platycladus - no smell Chamaecyparis Thuja - sweet pineapple

\hypertarget{key-1.3-young-twigs-brown-or-grey-1}{%
\subsection{Key 1.3: Young twigs brown or
grey}\label{key-1.3-young-twigs-brown-or-grey-1}}

\begin{enumerate}
\def\labelenumi{\arabic{enumi}.}
\tightlist
\item
  Leaves with sucker like base, attached directly to twig\ldots{}
  \emph{Abies}
\item
  Leaves without sucker like base, attached to a projection on twig
  -\textgreater{} 3

  \begin{enumerate}
  \def\labelenumii{\arabic{enumii}.}
  \setcounter{enumii}{2}
  \tightlist
  \item
    Leaves with indistinct petiole, twig very rough when leaves
    fallen\ldots{} \emph{Picea}
  \item
    Leaves with obvious petiole, bare shoots slightly rough
    -\textgreater{} 5
  \end{enumerate}
\item
  Leaves minutely serrate, buds hidden by leaves\ldots{} \emph{Tsuga}
\item
  Leaves entire, buds prominent\ldots{} \emph{Pseudotsuga} (menziesii)
\end{enumerate}

\hypertarget{key-1.4-young-twigs-green-1}{%
\subsection{Key 1.4: Young twigs
green}\label{key-1.4-young-twigs-green-1}}

\begin{enumerate}
\def\labelenumi{\arabic{enumi}.}
\tightlist
\item
  Leaves with a petiole\ldots{} \emph{Taxus} (baccata)
\item
  Leaves sessile -\textgreater{} 3

  \begin{enumerate}
  \def\labelenumii{\arabic{enumii}.}
  \setcounter{enumii}{2}
  \tightlist
  \item
    Leaves of two kinds, scale like, and linear/flat\ldots{}
    \emph{Sequoia} (sempervirens)
  \item
    Leaves all similar -\textgreater{} 5
  \end{enumerate}
\item
  Leaves \textgreater{}25mm long, \textgreater{}10mm wide, sharply spine
  tipped\ldots{} \emph{Araucaria} (araucana)
\item
  Leaves \textless{}25mm long -\textgreater{} 7

  \begin{enumerate}
  \def\labelenumii{\arabic{enumii}.}
  \setcounter{enumii}{6}
  \tightlist
  \item
    Leaves spirally arranged, with free part 3-7mm\ldots{}
    \emph{Sequoiadendron} (giganteum)
  \item
    Leaves in 5 ranks, incurved, with free part 5-20mm\ldots{}
    \emph{Cryptomeria} (japonica)
  \end{enumerate}
\end{enumerate}

\hypertarget{key-2-leaves-evergreen-parallel-veined-asparagaceae}{%
\section{Key 2: Leaves evergreen, parallel veined
(Asparagaceae)}\label{key-2-leaves-evergreen-parallel-veined-asparagaceae}}

\begin{enumerate}
\def\labelenumi{\arabic{enumi}.}
\tightlist
\item
  Leaves \textgreater{}20cm long in rosettes at apex of stem, palm-like
  -\textgreater{} 3
\item
  Leaves \textless{}20cm long spirally arranged on stems \ldots{}
  \emph{Ruscus}
\item
  Leaves with translucent veins \ldots{} \emph{Cordyline} (australis)
\item
  Leaves with indistinct opaque veins \ldots{} \emph{Yucca}
\end{enumerate}

\hypertarget{key-3-leaves-or-leaf-scars-whorled}{%
\section{Key 3: Leaves or leaf scars
whorled}\label{key-3-leaves-or-leaf-scars-whorled}}

\begin{enumerate}
\def\labelenumi{\arabic{enumi}.}
\tightlist
\item
  Leaves evergreen -\textgreater{} 3
\item
  Leaves deciduous, leaf scars apparent -\textgreater{} 15

  \begin{enumerate}
  \def\labelenumii{\arabic{enumii}.}
  \setcounter{enumii}{2}
  \tightlist
  \item
    Leaves pseudowhorled -\textgreater{} 5
  \item
    Leaves in true whorls -\textgreater{} 7
  \end{enumerate}
\item
  Matted evergreen shrub \textless{} 0.4m \ldots{} \emph{Empetrum}
\item
  Tall shrub to small tree \textgreater{}2m \ldots{} \emph{Rhododendron}

  \begin{enumerate}
  \def\labelenumii{\arabic{enumii}.}
  \setcounter{enumii}{6}
  \tightlist
  \item
    Leaves 3 whorled -\textgreater{} 9
  \item
    Leaves \textgreater{}4 whorled -\textgreater{} 11
  \end{enumerate}
\item
  Young twigs glandular hairy \ldots{} \emph{Erica}
\item
  Young twigs not glandular hairy -\textgreater{} 11

  \begin{enumerate}
  \def\labelenumii{\arabic{enumii}.}
  \setcounter{enumii}{10}
  \tightlist
  \item
    Leaves \textgreater{} 10mm wide \ldots{} \emph{Kalmia}
  \item
    Leaves \textless{} 1mm wide \ldots{} \emph{Erica}
  \end{enumerate}
\item
  Leaves tough \textgreater{}6mm wide, strongly retrorsely scabrid
  \ldots{} \emph{Rubia}
\item
  Lvs ± 1mm wide \ldots{} \emph{Erica}

  \begin{enumerate}
  \def\labelenumii{\arabic{enumii}.}
  \setcounter{enumii}{14}
  \tightlist
  \item
    Leaf scars minute, many, alternate, raised on spurs on second year
    shoots, reproductive organs cones \ldots{} \emph{Larix}
  \item
    Leaf scars larger, 3 per node, not raised -\textgreater{} 17
  \end{enumerate}
\item
  Twigs stout, leaf scars large with \textasciitilde{} 12 bundle traces
  in an ellipse. Buds usually arranged in two large and one small
  \ldots{} \emph{Catalpa} (bignonioides)
\item
  Twigs slender, leaf scars smaller with 3 or fewer bundle traces
  -\textgreater{} 19

  \begin{enumerate}
  \def\labelenumii{\arabic{enumii}.}
  \setcounter{enumii}{18}
  \tightlist
  \item
    Bundle trace 1, twigs mostly dead\ldots{} \emph{Fuschia}
  \item
    Bundle traces 3 -\textgreater{} 21
  \end{enumerate}
\item
  Twigs with stellate scales \ldots{} \emph{Deutzia} (better
  characters?)
\item
  Twigs hairless or with simple hairs \ldots{} \emph{Hydrangea}
\end{enumerate}

Add Philadelphus?

\hypertarget{key-4-leaves-or-leaf-scars-opposite}{%
\section{Key 4: Leaves or leaf scars
opposite}\label{key-4-leaves-or-leaf-scars-opposite}}

\begin{enumerate}
\def\labelenumi{\arabic{enumi}.}
\tightlist
\item
  Plants scrambling, climbing or parasitic on trees -\textgreater{} Key
  4.1
\item
  Plants not as above -\textgreater{} 3

  \begin{enumerate}
  \def\labelenumii{\arabic{enumii}.}
  \setcounter{enumii}{2}
  \tightlist
  \item
    Plants evergreen -\textgreater{} Key 4.2
  \item
    Plants deciduous -\textgreater{} Key 4.3
  \end{enumerate}
\end{enumerate}

\hypertarget{key-4.1-climbing-or-epiphytic-plants-parasitic-on-trees}{%
\subsection{Key 4.1: Climbing, or epiphytic plants parasitic on
trees}\label{key-4.1-climbing-or-epiphytic-plants-parasitic-on-trees}}

\begin{enumerate}
\def\labelenumi{\arabic{enumi}.}
\tightlist
\item
  Plant parasitic on tree branches with twigs repeatedly forked at the
  nodes -\textgreater{} 3
\item
  Plant rooted in soil -\textgreater{} 5

  \begin{enumerate}
  \def\labelenumii{\arabic{enumii}.}
  \setcounter{enumii}{2}
  \tightlist
  \item
    Twigs sickly green \ldots{} \emph{Viscum} (album)
  \item
    Twigs brown \ldots{} \emph{Loranthus} (rare hortal)
  \end{enumerate}
\item
  Evergreen (or semi-evergreen) -\textgreater{} 7
\item
  Deciduous -\textgreater{} 9

  \begin{enumerate}
  \def\labelenumii{\arabic{enumii}.}
  \setcounter{enumii}{6}
  \tightlist
  \item
    Leaves simple \ldots{} \emph{Lonicera}
  \item
    Leaves compound, climbing using coiling petioles \ldots{}
    \emph{Clematis}
  \end{enumerate}
\item
  Climbing by coiling petioles, persisting in winter \ldots{}
  \emph{Clematis}
\item
  Climbing by aerial roots or scrambling -\textgreater{} 11

  \begin{enumerate}
  \def\labelenumii{\arabic{enumii}.}
  \setcounter{enumii}{10}
  \tightlist
  \item
    Anvil shaped hairs on stems, stems twining clockwise, fruits
    persistent with many overlapping papery bracts \ldots{}
    \emph{Humulus} (lupulus)
  \item
    Not as above -\textgreater{} 13
  \end{enumerate}
\item
  Bundle traces 1, scrambling, weakly climbing \ldots{} \emph{Jasminum}
  (nudiflorum)
\item
  Bundle traces \textgreater{}1 -\textgreater{} 15

  \begin{enumerate}
  \def\labelenumii{\arabic{enumii}.}
  \setcounter{enumii}{14}
  \tightlist
  \item
    Bundle traces 3 \ldots{} \emph{Lonicera}
  \item
    Bundle traces 5 \ldots{} \emph{Schizophragma} (hydrangoides)
  \end{enumerate}
\end{enumerate}

\hypertarget{key-4.2-evergreen-plants}{%
\subsection{Key 4.2: Evergreen plants}\label{key-4.2-evergreen-plants}}

\begin{enumerate}
\def\labelenumi{\arabic{enumi}.}
\tightlist
\item
  Leaves simple and entire -\textgreater{} 3
\item
  Leaves compound or toothed -\textgreater{} 5

  \begin{enumerate}
  \def\labelenumii{\arabic{enumii}.}
  \setcounter{enumii}{2}
  \tightlist
  \item
    Trailing, thin, wiry shrubs \textless{}20cm tall at maturity
    -\textgreater{} Key 4.2.1
  \item
    Tree or upright shrub -\textgreater{} Key 4.2.2
  \end{enumerate}
\item
  Leaves ternate\ldots{} \emph{Choisya} (ternata)
\item
  Leaves toothed -\textgreater{} Key 4.2.3
\end{enumerate}

\hypertarget{key-4.2.1-wiring-creeping-trailing-shrubs}{%
\subsubsection{Key 4.2.1: Wiring, creeping, trailing
shrubs}\label{key-4.2.1-wiring-creeping-trailing-shrubs}}

\begin{enumerate}
\def\labelenumi{\arabic{enumi}.}
\tightlist
\item
  Leaves with stellate hairs \ldots{} \emph{Helianthemum}
\item
  Leaves with simple hairs or hairless -\textgreater{} 3

  \begin{enumerate}
  \def\labelenumii{\arabic{enumii}.}
  \setcounter{enumii}{2}
  \tightlist
  \item
    Leaves translucent dotted with sunken glands, aromatic \ldots{}
    \emph{Thymus}
  \item
    Leaves not translucent dotted or aromatic -\textgreater{} 5
  \end{enumerate}
\item
  Leaves with revolute margins -\textgreater{} 7
\item
  Leaves with flat margins -\textgreater{} 11

  \begin{enumerate}
  \def\labelenumii{\arabic{enumii}.}
  \setcounter{enumii}{6}
  \tightlist
  \item
    Leaves \textless{} 2mm long, petiole absent \ldots{} \emph{Calluna}
  \item
    Leaves \textgreater{}2mm long, petiole present -\textgreater{} 9
  \end{enumerate}
\item
  Leaves linear, rounded in cross section, 2-5mm long \ldots{}
  \emph{Frankenia}
\item
  Leaves oblong, 5-8mm long -\textgreater{} \emph{Loiseleuria}

  \begin{enumerate}
  \def\labelenumii{\arabic{enumii}.}
  \setcounter{enumii}{10}
  \tightlist
  \item
    Stipules absent -\textgreater{} 13
  \item
    Stipules present \ldots{} \emph{Herniaria}
  \end{enumerate}
\item
  \emph{Veronica} bifurcate Linnaea here
\item
  \emph{Veronica} bifurcate Linnaea here
\end{enumerate}

Linnaea here?

\hypertarget{key-4.2.2-shrubs-trees.-leaves-simple.}{%
\subsubsection{Key 4.2.2: Shrubs, trees. Leaves
simple.}\label{key-4.2.2-shrubs-trees.-leaves-simple.}}

\begin{enumerate}
\def\labelenumi{\arabic{enumi}.}
\tightlist
\item
  Leaves with clear 2 pinnate translucent veins on leaves
  -\textgreater{} 3
\item
  Leaves with 2 pinnate veins indistinct or absent -\textgreater{} 19

  \begin{enumerate}
  \def\labelenumii{\arabic{enumii}.}
  \setcounter{enumii}{2}
  \tightlist
  \item
    Leaves sessile -\textgreater{} 3
  \item
    Leaves petiolate -\textgreater{} 9
  \end{enumerate}
\item
  Leaves \textless{}1.5cm wide and with revolute margins at maturity
  -\textgreater{} 5
\item
  Leaves \textgreater{}2cm wide, with either transulcent dotted leaves
  or with minute glands attached to veinlets. Leaf scars triangular,
  bundle trace 1, pith spongey and excavated (hollow) -\textgreater{}
  \emph{Hypericum}
\item
  Leaves \textless{}4mm wide, rosemary scented. Bundle traces 3 \ldots{}
  \emph{Rosmarinus}
\item
  Leaves \textgreater{}4mm wide (on average), odourless. Bundle traces
  in a transverse line \ldots{} \emph{Kalmia}

  \begin{enumerate}
  \def\labelenumii{\arabic{enumii}.}
  \setcounter{enumii}{8}
  \tightlist
  \item
    Leaves \textgreater{}2cm long -\textgreater{} 11
  \item
    Leaves \textless{}2cm long \ldots{} \emph{Lonicera} (leaves
    \textless{}2cm long? Really?)
  \end{enumerate}
\item
  Leaves with dendritic hairs \ldots{} \emph{Phlomis}
\item
  Leaves with hairs simple or absent -\textgreater{} 13

  \begin{enumerate}
  \def\labelenumii{\arabic{enumii}.}
  \setcounter{enumii}{12}
  \tightlist
  \item
    Leaves with dense silver silky hairs below \ldots{} \emph{Olearia}
    (GARRYA? LEAVES GREY WOOLLY BELOW AND UNDULATE LEAF MARGINS?)
  \item
    Leaves hairless both sides, except for vein axils below
    -\textgreater{} 15
  \end{enumerate}
\item
  Vein axils below with tufts of hairs \ldots{} \emph{Viburnum}
\item
  Leaves glabrous -\textgreater{} 17

  \begin{enumerate}
  \def\labelenumii{\arabic{enumii}.}
  \setcounter{enumii}{16}
  \tightlist
  \item
    Sipules present, fused between petiole bases, young twigs ± square
    \ldots{} \emph{Coprosma}
  \item
    Stipules absent, young twigs rounded, at most angled slightly below
    the nodes \ldots{} \emph{Ligustrum}
  \end{enumerate}
\item
  Leaves white hairy at least below, strongly aromatic -\textgreater{}
  21
\item
  Leaves never white hairy, aromatic or not -\textgreater{} 23

  \begin{enumerate}
  \def\labelenumii{\arabic{enumii}.}
  \setcounter{enumii}{20}
  \tightlist
  \item
    Leaves hairy with long stellate hairs both sides, lavendar scented
    \ldots{} \emph{Lavandula}
  \item
    Leaves glabrous above, rosemary scented \ldots{} \emph{Rosmarinus}
  \end{enumerate}
\item
  Twigs square, leaves odorous -\textgreater{} 25
\item
  Twigs round, leaves odourless -\textgreater{} 29

  \begin{enumerate}
  \def\labelenumii{\arabic{enumii}.}
  \setcounter{enumii}{24}
  \tightlist
  \item
    Leaves gland pitted one or either side -\textgreater{} 25
  \item
    Leaves not gland pitted -\textgreater{} \emph{Buxus}
  \end{enumerate}
\item
  Young twigs green, minutely ciliate, menthol scented \ldots{}
  \emph{Hyssopus}
\item
  Young twigs whitish, leaves long ciliate at base, sage scented
  \ldots{} \emph{Satureja}

  \begin{enumerate}
  \def\labelenumii{\arabic{enumii}.}
  \setcounter{enumii}{28}
  \tightlist
  \item
    Leaves strongly revolute, \textless{}2mm long, sessile \ldots{}
    \emph{Calluna}
  \item
    Leaves not revolute -\textgreater{} 31
  \end{enumerate}
\item
  Leaves fleshy, mealy grey \textless{}6cm, not valvate when developing
  \ldots{} \emph{Atriplex}
\item
  Leaves not fleshy, valvate when developing \ldots{} \emph{Veronica}
  (sect Hebe)
\end{enumerate}

\hypertarget{key-4.2.3-shrubs-trees.-leaves-toothed.}{%
\subsubsection{Key 4.2.3: Shrubs, trees. Leaves
toothed.}\label{key-4.2.3-shrubs-trees.-leaves-toothed.}}

\begin{enumerate}
\def\labelenumi{\arabic{enumi}.}
\tightlist
\item
  Twigs with dendritic hairs \ldots{} \emph{Phlomis}
\item
  Twigs never with dendritic hairs -\textgreater{} 3

  \begin{enumerate}
  \def\labelenumii{\arabic{enumii}.}
  \setcounter{enumii}{2}
  \tightlist
  \item
    Twigs with stellate hairs -\textgreater{} 5
  \item
    Twigs without stellate hairs -\textgreater{} 7
  \end{enumerate}
\item
  Stipules present, leaves white felted below \ldots{} \emph{Buddleja}
\item
  Stipules absent, leaves stellate hairy below \ldots{} \emph{Viburnum}

  \begin{enumerate}
  \def\labelenumii{\arabic{enumii}.}
  \setcounter{enumii}{6}
  \tightlist
  \item
    Leaves \textless{} 2cm long, stems creeping and prostrate \ldots{}
    \emph{Linnaea}
  \item
    Leaves \textgreater{}2cm long, stems never creeping -\textgreater{}
    9
  \end{enumerate}
\item
  Petiole with 1 vascular bundle -\textgreater{} 11
\item
  Petiole with 3 vascular bundles -\textgreater{} 13

  \begin{enumerate}
  \def\labelenumii{\arabic{enumii}.}
  \setcounter{enumii}{10}
  \tightlist
  \item
    Twigs square \ldots{} \emph{Phygelius}
  \item
    Twigs round -\textgreater{} 15
  \end{enumerate}
\item
  Not net veined, weak 2 pinnate veins. Leaves with yellow blotches
  above, petiole green \ldots{} \emph{Aucuba}
\item
  Net veined and strongly 3-pli-veined. No yellow blotches, petiole
  reddish \ldots{} \emph{Viburnum}

  \begin{enumerate}
  \def\labelenumii{\arabic{enumii}.}
  \setcounter{enumii}{14}
  \tightlist
  \item
    Young twigs brown, each leaf tooth with a fragile claw like gland
    \ldots{} \emph{Euonymus}
  \item
    Young twigs green, leaf teeth without glands -\textgreater{} 17
  \end{enumerate}
\item
  Leaves cuneate at base, buds with scales \ldots{} \emph{Rhamnus}
\item
  Leaves rounded at base, buds naked \ldots{} \emph{Phillyrea}
\end{enumerate}

Evergreen:

TOOTHED LVS Aucuba - bt's 3, pith chambered with granular septa, scars
slightly raised, large and crescent shaped

Buddleja - stems 4 ridged, pith large, white and continuous, scars half
round or triangular, small, bt 1. Stipule scars transversely connected
Viburnum - bt's 3 or in groups of 3 with the middle bundles 2, pith
continuous, leaf scars crescent shaped or angular v shaped Linnaea -
leaf scars raised and shrivelled, single bt obscured Euonymus - scars
small, elevated, bt 1 towards the top of the scar Phygelius - twigs
square Rhamnus - pith rounded continuous and white, bt's 3 or in a
transverse series, buds with scales Phillyrea - bt 1, scars crescent
shaped

LVS TERNATE Choisya - ! ENTIRE LVS Phlomis - Coprosma - Buxus - bt 1,
leaf scars minute and crescent shaped, pith minute and cont. Twigs flat
grooved between each pair of leaves Olearia Kalmia - bt's in a
transverse line, scars half round or shield shaped, pith small and cont
Calluna - scars opposite, minute, crescent shaped, bt 1, pith round cont
Hebe - Viburnum - bt's 3 or in groups of 3 with the middle bundles 2,
pith continuous, leaf scars crescent shaped or angular v shaped
Ligustrum - pith moderate, white and homogenous, scars crescent shaped
with bt 1 Lavandula - Rosmarinus - twigs 4 sided, leaf scars deeply u
shaped, bt's 3 Hypericum - leaf scars triangular, 1 bt, pith spongey and
finally excavated ?? Garrya - bt's 3, pith continuous, scars angularly
u-shaped

?? Hyssopus Satureja ?? Atriplex

?? Helianthemum Thymus Loiseleuria Frankenia Herniaria Fuschia Weigela
Kolkwitzia Phygelius Phillyrea ?? Akebia

\hypertarget{key-4.3-deciduous-plants}{%
\subsection{Key 4.3: Deciduous plants}\label{key-4.3-deciduous-plants}}

\begin{enumerate}
\def\labelenumi{\arabic{enumi}.}
\tightlist
\item
  Bundle trace 1 -\textgreater{} Key 4.3.1
\item
  Bundle traces \textgreater{}1 -\textgreater{} 5

  \begin{enumerate}
  \def\labelenumii{\arabic{enumii}.}
  \setcounter{enumii}{2}
  \tightlist
  \item
    Bundle traces 3 -\textgreater{} Key 4.3.2
  \item
    Bundle traces \textgreater{}3 -\textgreater{} Key 4.3.3
  \end{enumerate}
\end{enumerate}

\hypertarget{key-4.3.1-bundle-trace-1}{%
\subsubsection{Key 4.3.1: Bundle trace
1}\label{key-4.3.1-bundle-trace-1}}

\begin{enumerate}
\def\labelenumi{\arabic{enumi}.}
\tightlist
\item
  Twigs green \ldots{} \emph{Euonymus}
\item
  Twigs not green when mature -\textgreater{} 3

  \begin{enumerate}
  \def\labelenumii{\arabic{enumii}.}
  \setcounter{enumii}{2}
  \tightlist
  \item
    Trees -\textgreater{} 5
  \item
    Shrubs -\textgreater{} 9
  \end{enumerate}
\item
  Terminal buds always present, buds felted brown, grey or black
  \ldots{} \emph{Fraxinus}
\item
  Terminal buds absent -\textgreater{} 11

  \begin{enumerate}
  \def\labelenumii{\arabic{enumii}.}
  \setcounter{enumii}{6}
  \tightlist
  \item
    Pith chambered, twigs densly hairy \ldots{} \emph{Paulownia}
    (tomentosa)
  \item
    Pith solid, twigs red brown\ldots{} \emph{Metasequoia}
    (glyptostryboides)
  \end{enumerate}
\item
  Buds sometimes whorled in 3's\ldots{} \emph{Fuschia}
\item
  Buds strictly opposite -\textgreater{} 11

  \begin{enumerate}
  \def\labelenumii{\arabic{enumii}.}
  \setcounter{enumii}{10}
  \tightlist
  \item
    Pith present, solid, whitish -\textgreater{} 13
  \item
    Twigs hollow or chambered, brownish -\textgreater{} 15
  \end{enumerate}
\item
  Buds mostly in pairs at twig apices, glandular ciliate with 6-7 scales
  \ldots{} \emph{Syringa}
\item
  Buds singular at twig apices, ciliate with ± 4 pairs of opposite
  scales \ldots{} \emph{Ligustrum}

  \begin{enumerate}
  \def\labelenumii{\arabic{enumii}.}
  \setcounter{enumii}{14}
  \tightlist
  \item
    Buds \textgreater{}11 pairs of scales, twigs chambered to hollow
    between nodes \ldots{} \emph{Forsythia}
  \item
    Buds \textless{}4 pairs of scales, or absent, twigs hollow but never
    chambered -\textgreater{} 17
  \end{enumerate}
\item
  Leaf scars mostly torn, partly connected by transverse ridges, buds
  with keeled scales. Fruits white or red berries \ldots{}
  \emph{Symphoricarpos}
\item
  Leaf scars never torn, triangularly lens shaped, not partly connected
  by ridges, buds without scales \ldots{} \emph{Hypericum}
\end{enumerate}

\hypertarget{key-4.3.2-bundle-traces-3}{%
\subsubsection{Key 4.3.2: Bundle traces
3}\label{key-4.3.2-bundle-traces-3}}

\begin{enumerate}
\def\labelenumi{\arabic{enumi}.}
\tightlist
\item
  Transverse ridge not present between buds or leaf scars
  -\textgreater{} 3
\item
  Transverse ridge present between buds or leaf scars, or leaf scars
  abut -\textgreater{} 7

  \begin{enumerate}
  \def\labelenumii{\arabic{enumii}.}
  \setcounter{enumii}{2}
  \tightlist
  \item
    Shoots thorny, buds with 8-10 dark brown scales, terminal buds
    present\ldots{} \emph{Rhamnus} (cathartica)
  \item
    Shoots not thorny, buds red with one scale visible, terminal buds
    absent\ldots{} \emph{Cercidiphyllum} (japonicum)
  \end{enumerate}
\item
  Buds hidden behind leaf scar (which is like a thin membrane)\ldots{}
  \emph{Philadelphus}
\item
  Buds exposed and obvious -\textgreater{} 7

  \begin{enumerate}
  \def\labelenumii{\arabic{enumii}.}
  \setcounter{enumii}{6}
  \tightlist
  \item
    Buds naked -\textgreater{} 9
  \item
    Buds with scales -\textgreater{} 13
  \end{enumerate}
\item
  Only terminal bud naked, twigs stout\ldots{} \emph{Hydrangea}
\item
  All buds naked -\textgreater{} 11

  \begin{enumerate}
  \def\labelenumii{\arabic{enumii}.}
  \setcounter{enumii}{10}
  \tightlist
  \item
    Twigs but especially buds with stellate hairs\ldots{}
    \emph{Viburnum}
  \item
    Twigs and buds without stellate hairs but with medifixed
    hairs\ldots{} \emph{Cornus}
  \end{enumerate}
\item
  Stipule scars large and obvious\ldots{} \emph{Staphylea}
\item
  Stipule scars absent or not obvious -\textgreater{} 15

  \begin{enumerate}
  \def\labelenumii{\arabic{enumii}.}
  \setcounter{enumii}{14}
  \tightlist
  \item
    Buds with a pair of scales fused, enveloping bud, buds globose, red.
    Or flowering in winter, fragrant\ldots{} \emph{Viburnum}
  \item
    Bud scales \textgreater{}2, not fused, not flowering in winter
    -\textgreater{} 17
  \end{enumerate}
\item
  Pith hollow between nodes -\textgreater{} 19
\item
  Pith solid, or spongey -\textgreater{} 23

  \begin{enumerate}
  \def\labelenumii{\arabic{enumii}.}
  \setcounter{enumii}{18}
  \tightlist
  \item
    Buds solitary, twigs green \ldots{} \emph{Leycesteria}
  \item
    Buds often superposed or collateral, twigs never green
    -\textgreater{} 21
  \end{enumerate}
\item
  Buds \textgreater{}3mm long, with buds often superposed\ldots{}
  \emph{Lonicera}
\item
  Buds \textless{}3mm long, buds often collateral\ldots{}
  \emph{Symphoricarpos}

  \begin{enumerate}
  \def\labelenumii{\arabic{enumii}.}
  \setcounter{enumii}{22}
  \tightlist
  \item
    Twigs with 2-4 ridges decurrent from nodes. 2-valved capsules
    persistent \ldots{} \emph{Weigela}
  \item
    Twigs without ridges decurrent from nodes, fruits rarely persistent
    -\textgreater{} 25
  \end{enumerate}
\item
  Bundle traces forming a line, fruits persistent, bristly with 5 lobed
  calyx\ldots{}\emph{Kolkwitzia}
\item
  Bundle traces distinct, fruits not persistent, or if they are, not as
  above -\textgreater{} 27

  \begin{enumerate}
  \def\labelenumii{\arabic{enumii}.}
  \setcounter{enumii}{26}
  \tightlist
  \item
    Pith spongey, twigs stout\ldots{} \emph{Sambucus}
  \item
    Pith solid, twigs more slender\ldots{} \emph{Acer}
  \end{enumerate}
\end{enumerate}

\hypertarget{key-4.3.3-bundle-traces-3}{%
\subsubsection{Key 4.3.3: Bundle traces
\textgreater{}3}\label{key-4.3.3-bundle-traces-3}}

\begin{enumerate}
\def\labelenumi{\arabic{enumi}.}
\tightlist
\item
  Leaf bases obscuring leaf scars -\textgreater{} 3
\item
  Leaf scars clearly present -\textgreater{} 5

  \begin{enumerate}
  \def\labelenumii{\arabic{enumii}.}
  \setcounter{enumii}{2}
  \tightlist
  \item
    Twigs green, hollow\ldots{} \emph{Leycesteria} (formosana)
  \item
    Twigs red or purplish, solid\ldots{} \emph{Acer} (section Palmata)
  \end{enumerate}
\item
  At least some nodes in whorls of 3 -\textgreater{} 7
\item
  Nodes strictly opposite -\textgreater{} 11

  \begin{enumerate}
  \def\labelenumii{\arabic{enumii}.}
  \setcounter{enumii}{6}
  \tightlist
  \item
    Bundle traces \textgreater{}8 in an ellipse or horseshoe shape,
    terminal bud absent -\textgreater{} 9
  \item
    Bundle traces \textless{} 7 (usually 3-5), terminal bud
    present\ldots{} \emph{Hydrangea}
  \end{enumerate}
\item
  Each node with one small leaf scar, and two large\ldots{}
  \emph{Catalpa} (bignonioides)
\item
  Leaf scars all the same size\ldots{} \emph{Clerodendron} (trichotomum)

  \begin{enumerate}
  \def\labelenumii{\arabic{enumii}.}
  \setcounter{enumii}{10}
  \tightlist
  \item
    Large stipule scars between leaf scars\ldots{} \emph{Staphylea}
    (pinnata)
  \item
    If stipule scars present, inconspicuous -\textgreater{} 13
  \end{enumerate}
\item
  Trees -\textgreater{} 15
\item
  Shrubs -\textgreater{} 21

  \begin{enumerate}
  \def\labelenumii{\arabic{enumii}.}
  \setcounter{enumii}{14}
  \tightlist
  \item
    Terminal bud present, pith solid -\textgreater{} 17
  \item
    Terminal bud absent, pith chambered\ldots{} \emph{Paulownia}
    (tomentosa)
  \end{enumerate}
\item
  Bundle traces many in a closed circle, buds furry\ldots{}
  \emph{Fraxinus}
\item
  Bundle traces \textless{}9, distinct, buds not furry but may be hairy
  -\textgreater{} 19

  \begin{enumerate}
  \def\labelenumii{\arabic{enumii}.}
  \setcounter{enumii}{18}
  \tightlist
  \item
    Terminal buds \textgreater{}15mm, leaf scars large, shield
    shaped\ldots{} \emph{Aesculus}
  \item
    Terminal buds \textless{} 15mm, leaf scars smaller, mostly crescent
    shaped\ldots{} \emph{Acer}
  \end{enumerate}
\item
  Terminal bud very large (15-20mm), naked\ldots{} \emph{Hydrangea}
\item
  Terminal bud smaller \textless{}15mm, often absent from shoots
  -\textgreater{} 23

  \begin{enumerate}
  \def\labelenumii{\arabic{enumii}.}
  \setcounter{enumii}{22}
  \tightlist
  \item
    Pith wide, spongey, twigs stout \ldots{} \emph{Sambucus}
  \item
    Pith narrower, hard, more slender \ldots{} \emph{Acer}
  \end{enumerate}
\end{enumerate}

Deciduous:

Fuschia Weigela Kolkwitzia

\begin{verbatim}
Clerodendron - twigs obscurely 4 sided, pith large round white and continuous, buds superposed but lower often obscured by leaf scar, round ovoid. Leaf scars sometimes in whorls of three, elliptical, bt's 9 or aggregated in a U shaped series (?). No interpetiolar ridge.
\end{verbatim}

-- Clerodendron trichotomum

\begin{verbatim}
Paulownia - twigs stout, compressed at nodes. Pith large and chambered or hollowed between nodes. Buds superposed and sessile with 4 exposed blunt scales. Scars elliptical and more or less notched at the top. Bt's many in a near closed elliptical series. Without interpetiolar ridge.
\end{verbatim}

QA - Paulownia tomentosa

\begin{verbatim}
Rhamnus - pith rounded, continuous and white, scars small or more or less raised, bt's 3 or in a transverse series, buds small sessile (naked or scaled). No interpetiolar ridge
\end{verbatim}

OB - Rhamnus alaternus

\begin{verbatim}
Aesculus - pith large, 6 sided, pale. Upper most buds very large, half dozen exposed scales. leaf scars shield shaped or triangular, bundle traces 3 or in three groups. Weakly ridged between buds
\end{verbatim}

RG - Aesculus hippocastanum RG - Aesculus carnea RG - Aesculus indica

\begin{verbatim}
      Sambucus - twigs (6)8-10 sided, pith large, soft and continuous, buds with 3-5 pairs of scales, end bud mostly lacking. Scars crescent shaped, 3-4 sided large and transversely connected. Bundle traces 3,5,7. Interpetiolar ridge present
\end{verbatim}

SJ - ebulus SC - nigra SC - racemosa

\begin{verbatim}
Fraxinus - twigs stout, stiff, squarish and compressed at nodes. Buds sessile, superposed, 2-3 pairs of opposite scales, those of end bud often lobed. Bt's in a u-shaped series. Scars half round and broadly u shaped. Without interpetiolar ridge.
\end{verbatim}

SC - excelsior SC - angustifolium SC - ornus

\begin{verbatim}
Viburnum (lantana, opulus) - twigs more or less 6 sided, pith round to 6 sided, continuous. End bud naked and stellate scurfy (lantana) or scaled and stalked ovoid (opulus). Scars crescent shaped, bt's 3
\end{verbatim}

QA - opulus (interpetiolar ridge) OA - lantana (interpetiolar ridge)

Leycesteria - soft wooded shrubs, twigs slender and round and pith
excavated. Buds with 1-2 pairs of exposed scales, the outer attenuate.
Leaf scars raised and minute so as to equal bud. Opposite scars connect
in a cross line. Bt's 3. With interpetiolar ridge

OA/LA - formosana

\begin{verbatim}
Cornus - twigs often bright coloured, round to 6 sided, pith white, continuous or soft or spongey. Buds stalked, oblong with a pair of valvate scales. Leaf scars raised and crescent shaped (raised on first winter on petiole bases). Bt's 3, leaf scars transversely joined. With interpetiolar ridge weak/absent
\end{verbatim}

LA - koenigii LA - mas LA - sanguinea LA - sericea

\begin{verbatim}
Deutzia -  twigs stellate pubescent, pith round, pale and spongey or brown and excavated between nodes (in our species yes). Buds with 2-6 pairs of scales. Scars triangular or transversely elongated and connected by transverse ridges. With interpetiolar ridge
\end{verbatim}

OA - scabra

Philadelphus - twigs lined to hexagonal, buds with 2 valvate and hairy
scales. Leaf scars half round with a thin membrane that covers the bud
or crescent shaped when this is burst, connected transversely. Bt's 3.
With interpetiolar ridge

OA - coronarius OA - microphyllus

Acer - twigs terete to six sided, pith round continuous and pale, buds
with 2 to several pairs of scales, leaf scars U shaped bundle traces 3,
5, 7, 9

QA - campestre QA - cappadocicum RG/SC - negundo QA - platanoides QA -
pseudoplatanus QA - saccharum

Lonicera - twigs rounded, slender, bt's 3. With interpetiolar ridge

LA - xylosteum LA - maackii LA - formosa LA - x purpusii LA - tatarica

Hydrangea - pith large, continuous and pale, buds with 4-6 exposed
scales, leaf scars crescent shaped, with connecting crossline. Bt's 3.
With interpetiolar ridge

OA - macrophylla

*Euonymus - twigs characteristically green, pith round to 4 angled,
greenish and spongey. Buds with 3-5 pairs of at first serrulate scales,
scars small, half elliptical, somewhat elevated. Bt 1, transverse
towards the top of the scar. No interpetiolar ridge.

OB - latifolius OB - europaeus

*Ligustrum - twigs rounded to 4 lined below the nodes, pith moderate
white and homogenous, buds with 2-3 exposed scales. Leaf scars crescent
shaped, raised and small. Bt 1. No interpetiolar ridge

LB - vulgare LB - ovalifolium (mostly evergreen)

*Forsythia - pith chambered to excavated, twigs moderately 4 sided. Buds
with a dozen pairs of scales. Leaf scars shield shaped, small. Bt 1. No
interpetiolar ridge

OB - x intermedia

*Symphoricarpos - small deciduous shrubs. Pith brownish, usually
excavated, buds with about 3 pairs of keeled scales. Leaf scars small
and mostly torn, raised and partly connected by transverse ridges, bt 1

LA - albus LA -orbiculatus LA - x chenaultii

\emph{Hypericum - twigs angled at least below the nodes, pith brown,
spongey and finally excavated. Buds with 2 or several scarcely
specialised scales. Scars triangularly lens shaped, bt 1 }Syringa -
twigs moderate to slender, pith homogenous, buds with around 4 pairs of
scales. Scars crescent or shield shaped. Bt 1 transverse and compound.
No interpetiolar ridge.

LB - vulgaris

\begin{verbatim}
      Staphylea - twigs moderate and rounded, glabrous. Pith large, continuous. Buds ovoid, glabrous 2 edged. Scars crescent shaped. Bt's 5 or 7. Stipule scars half round or elongated. With interpetiolar ridge.
\end{verbatim}

\hypertarget{key-5-leaves-or-leaf-scars-alternate}{%
\section{Key 5: Leaves or leaf scars
alternate}\label{key-5-leaves-or-leaf-scars-alternate}}

\begin{enumerate}
\def\labelenumi{\arabic{enumi}.}
\tightlist
\item
  Plants scrambling or climbing -\textgreater{} Key 5.1
\item
  Plants not as above -\textgreater{} 3

  \begin{enumerate}
  \def\labelenumii{\arabic{enumii}.}
  \setcounter{enumii}{2}
  \tightlist
  \item
    Plants evergreen -\textgreater{} Key 5.2
  \item
    Plants deciduous -\textgreater{} Key 5.3
  \end{enumerate}
\end{enumerate}

\hypertarget{key-5.1-plants-scrambling-or-climbing}{%
\subsection{Key 5.1: Plants scrambling or
climbing}\label{key-5.1-plants-scrambling-or-climbing}}

\begin{enumerate}
\def\labelenumi{\arabic{enumi}.}
\tightlist
\item
  Plant with leaf opposed tendrils -\textgreater{} 3
\item
  Plant without tendrils -\textgreater{} 5

  \begin{enumerate}
  \def\labelenumii{\arabic{enumii}.}
  \setcounter{enumii}{2}
  \tightlist
  \item
    Tendrils simple or forked and never thickened into pads, no
    lenticels on twigs, buds smaller, subglobose \ldots{} \emph{Vitis}
  \item
    Tendrils branched 2 or more times and often thickened into pads,
    lenticels on twigs, buds larger, round conical \ldots{}
    \emph{Parthenocissus}
  \end{enumerate}
\item
  Leaves evergreen -\textgreater{} 7
\item
  Leaves deciduous -\textgreater{} 9

  \begin{enumerate}
  \def\labelenumii{\arabic{enumii}.}
  \setcounter{enumii}{6}
  \tightlist
  \item
    Stems with adventitious aerial roots, leaves lobed\ldots{}
    \emph{Hedera}
  \item
    Without aerial adventitious roots, leaves almost always
    entire\ldots{} \emph{Muehlenbeckia}
  \end{enumerate}
\item
  Stipule (ochreae) scars encircling twigs, at least in older
  twigs\ldots{} \emph{Fallopia}
\item
  Stipule scars absent -\textgreater{} 11

  \begin{enumerate}
  \def\labelenumii{\arabic{enumii}.}
  \setcounter{enumii}{10}
  \tightlist
  \item
    Buds absent\ldots{} \emph{Salpichroa} (rare)
  \item
    Buds present -\textgreater{} 13
  \end{enumerate}
\item
  Buds superposed (one above another)\ldots{} \emph{Aristolochia}
\item
  Buds single at nodes -\textgreater{} 15

  \begin{enumerate}
  \def\labelenumii{\arabic{enumii}.}
  \setcounter{enumii}{14}
  \tightlist
  \item
    Bundle trace 1, buds with \textless{}\textless{} 12 scales
    -\textgreater{} 17
  \item
    Bundle traces 6 or 3 at level of stem, buds with \textasciitilde{}12
    mucronate scales\ldots{} \emph{Akebia}
  \end{enumerate}
\item
  Twigs pale, inconspicuously 5 ridged, hard wooded scrambling shrub,
  sometimes spiny\ldots{} \emph{Lycium}
\item
  Twigs darker, rounded, soft wooded twiner, never spiny. Frequently
  with dried berries in panicles\ldots{} \emph{Solanum}
\end{enumerate}

\hypertarget{key-5.2-evergreen-plants}{%
\subsection{Key 5.2: Evergreen plants}\label{key-5.2-evergreen-plants}}

\begin{enumerate}
\def\labelenumi{\arabic{enumi}.}
\tightlist
\item
  Leaves simple -\textgreater{} 3
\item
  Leaves compound -\textgreater{} Key 5.2.1

  \begin{enumerate}
  \def\labelenumii{\arabic{enumii}.}
  \setcounter{enumii}{2}
  \tightlist
  \item
    Leaves entire -\textgreater{} Key 5.2.2
  \item
    Leaves toothed or lobed -\textgreater{} Key 5.2.3
  \end{enumerate}
\end{enumerate}

\hypertarget{key-5.2.1-leaves-compound}{%
\subsubsection{Key 5.2.1: Leaves
compound}\label{key-5.2.1-leaves-compound}}

\begin{enumerate}
\def\labelenumi{\arabic{enumi}.}
\tightlist
\item
  Leaflets with spiny margins\ldots{} \emph{Mahonia}
\item
  Leaflets without spiny margins -\textgreater{} 3

  \begin{enumerate}
  \def\labelenumii{\arabic{enumii}.}
  \setcounter{enumii}{2}
  \tightlist
  \item
    Leaves 1-pinnate\ldots{} \emph{Coronilla}
  \item
    Leaves \textgreater{} 1 pinnate\ldots{} \emph{Acacia}
  \end{enumerate}
\end{enumerate}

\hypertarget{key-5.2.2-leaves-entire}{%
\subsubsection{Key 5.2.2: Leaves entire}\label{key-5.2.2-leaves-entire}}

\begin{enumerate}
\def\labelenumi{\arabic{enumi}.}
\tightlist
\item
  Tree or tall shrub \textgreater{}1.2m -\textgreater{} 3
\item
  Low growing shrub \textless{}1.2m, prostrate, erect or mat forming
  -\textgreater{} 35

  \begin{enumerate}
  \def\labelenumii{\arabic{enumii}.}
  \setcounter{enumii}{2}
  \tightlist
  \item
    Leaves with peltate scales or stellate hairs -\textgreater{} 5
  \item
    Leaves without peltate scales or stellate hairs -\textgreater{} 9
  \end{enumerate}
\item
  Peltate scales present -\textgreater{} 7
\item
  Peltate scales absent, stellate hairs present. Leaves spine tipped
  when young -\textgreater{} \emph{Quercus}

  \begin{enumerate}
  \def\labelenumii{\arabic{enumii}.}
  \setcounter{enumii}{6}
  \tightlist
  \item
    Leaves strongly net veined\ldots{} \emph{Olearia}
  \item
    Leaves not strongly net veined\ldots{}\emph{Eleagnus}
  \end{enumerate}
\item
  Stipules modified into spines (in leaf axils), wood yellow
  underneath\ldots{} \emph{Berberis}
\item
  Spines not present -\textgreater{} 11

  \begin{enumerate}
  \def\labelenumii{\arabic{enumii}.}
  \setcounter{enumii}{10}
  \tightlist
  \item
    Leaves white felted or woolly beneath -\textgreater{} 13
  \item
    Leaves never white felted -\textgreater{} 15
  \end{enumerate}
\item
  Leaves white felted, buds naked\ldots{} \emph{Brachyglottis}
\item
  Leaves short woolly below, bud with several scales\ldots{}
  \emph{Pittosporum}

  \begin{enumerate}
  \def\labelenumii{\arabic{enumii}.}
  \setcounter{enumii}{14}
  \tightlist
  \item
    Petiole with sheathing base on stem, twigs yellowish green\ldots{}
    \emph{Griselinia}
  \item
    Petiole never sheathing, twigs green or brown -\textgreater{} 17
  \end{enumerate}
\item
  Twigs and leaves with almond odour when crushed\ldots{}
  \emph{Photinia}
\item
  Not with an almond odour -\textgreater{} 19

  \begin{enumerate}
  \def\labelenumii{\arabic{enumii}.}
  \setcounter{enumii}{18}
  \tightlist
  \item
    Leaves bay scented\ldots{} \emph{Laurus}
  \item
    Leaves not bay scented -\textgreater{} 21
  \end{enumerate}
\item
  Leaves dotted with translucent glands\ldots{} \emph{Skimmia}
\item
  Leaves never with translucent glands -\textgreater{} 23

  \begin{enumerate}
  \def\labelenumii{\arabic{enumii}.}
  \setcounter{enumii}{22}
  \tightlist
  \item
    Leaves sessile\ldots{} \emph{Bupleurum}
  \item
    Petiole present -\textgreater{} 25
  \end{enumerate}
\item
  Leaves spine tipped at apex\ldots{} \emph{Ilex}
\item
  Leaves never spine tipped -\textgreater{} 27

  \begin{enumerate}
  \def\labelenumii{\arabic{enumii}.}
  \setcounter{enumii}{26}
  \tightlist
  \item
    Leaves sivery due to crystalline cells on both sides\ldots{}
    \emph{Atriplex}
  \item
    Leaves never silvery -\textgreater{} 29
  \end{enumerate}
\item
  Twigs hairless -\textgreater{} 31
\item
  Twigs hairy (look closely) -\textgreater{} 33
\item
  Leaves much larger \textgreater{} 8cm long\ldots{} \emph{Rhododendron}
\item
  Leaves smaller \textless{} 7cm\ldots{} \emph{Pittosporum}
\item
  Leaves in pseudowhorls at twig apex leading to bunched leaf scars from
  last years growth, twig hairs not S-shaped at base. Stipules
  absent\ldots{} \emph{Pittosporum}
\item
  Leaves never in pseudowhorls, twig hairs S-shaped at base, stipules
  present and soon falling\ldots{} \emph{Cotoneaster} 35 (2). Leaves
  revolute at least when young, often strongly revolute at maturity
  -\textgreater{} 37

  \begin{enumerate}
  \def\labelenumii{\arabic{enumii}.}
  \setcounter{enumii}{35}
  \tightlist
  \item
    Leaves with flat margins, not revolute when young -\textgreater{} 47
  \end{enumerate}
\item
  Leaves hairless -\textgreater{} 39
\item
  Leaves hairy -\textgreater{} 45

  \begin{enumerate}
  \def\labelenumii{\arabic{enumii}.}
  \setcounter{enumii}{38}
  \tightlist
  \item
    Leaves strongly revolute, cylindrical, glandular ciliate only when
    young \ldots{} \emph{Empetrum}
  \item
    Leaves weakly revolute -\textgreater{} 41
  \end{enumerate}
\item
  Leaves gland dotted below\ldots{} \emph{Vaccinium}
\item
  Leaves not gland dotted below -\textgreater{} 43

  \begin{enumerate}
  \def\labelenumii{\arabic{enumii}.}
  \setcounter{enumii}{42}
  \tightlist
  \item
    Stems \textgreater{}1mm diameter, not rooting at nodes\ldots{}
    \emph{Andromeda}
  \item
    Stems \textless{}1mm diameter, rooting at nodes\ldots{}
    \emph{Vaccinium}
  \end{enumerate}
\item
  Leaves orange woolly below\ldots{} \emph{Rhododendron} (subgenus
  Ledum)
\item
  Leaves white woolly below\ldots{} \emph{Daboecia}

  \begin{enumerate}
  \def\labelenumii{\arabic{enumii}.}
  \setcounter{enumii}{46}
  \tightlist
  \item
    Stipules or stipule scars present\ldots{} \emph{Cotoneaster}
    (Herniaria keys out here if evergreen, but apparently alternate
    leaved due to one leaf smaller or aborted)
  \item
    Stipules absent -\textgreater{} 49
  \end{enumerate}
\item
  Leaves fleshy -\textgreater{} 51
\item
  Leaves not fleshy -\textgreater{} 55

  \begin{enumerate}
  \def\labelenumii{\arabic{enumii}.}
  \setcounter{enumii}{50}
  \tightlist
  \item
    Leaves cylindrical, or nearly so -\textgreater{} 53
  \item
    Leaves flat, green\ldots{} \emph{Sedum}
  \end{enumerate}
\item
  Leaves glaucous\ldots{} \emph{Suaeda}
\item
  Leaves green\ldots{} \emph{Artemisia}

  \begin{enumerate}
  \def\labelenumii{\arabic{enumii}.}
  \setcounter{enumii}{54}
  \tightlist
  \item
    Leaves ± linear\ldots{} \emph{Iberis}
  \item
    Leaves more obovate -\textgreater{} 57
  \end{enumerate}
\item
  Leaves \textgreater{}5cm\ldots{} \emph{Daphne}
\item
  Leaves \textless{}3cm -\textgreater{} 59

  \begin{enumerate}
  \def\labelenumii{\arabic{enumii}.}
  \setcounter{enumii}{58}
  \tightlist
  \item
    Leaves strongly net veined\ldots{} \emph{Arctostaphylos}
  \item
    Leaves not net veined, very rare\ldots{} \emph{Diapensia}
  \end{enumerate}
\end{enumerate}

\hypertarget{key-5.2.3-leaves-toothed-or-lobed}{%
\subsubsection{Key 5.2.3: Leaves toothed or
lobed}\label{key-5.2.3-leaves-toothed-or-lobed}}

\begin{enumerate}
\def\labelenumi{\arabic{enumi}.}
\tightlist
\item
  Leaves toothed -\textgreater{} 3
\item
  Leaves lobed -\textgreater{} 39

  \begin{enumerate}
  \def\labelenumii{\arabic{enumii}.}
  \setcounter{enumii}{2}
  \tightlist
  \item
    Branches spiny or thorny -\textgreater{} 5
  \item
    Branches unarmed -\textgreater{} 7
  \end{enumerate}
\item
  Spines at stem nodes, leaves spine toothed\ldots{} \emph{Berberis}
\item
  Spines at ends of branches, leaves not spine toothed\ldots{}
  \emph{Pyracantha} (coccinea)
\item
  Low shrub \textless{} 60cm tall -\textgreater{} 9
\item
  Large shrub or tree -\textgreater{} 17
\item
  Leaves strongly net veined -\textgreater{} 11
\item
  Leaves not net veined -\textgreater{} 13

  \begin{enumerate}
  \def\labelenumii{\arabic{enumii}.}
  \setcounter{enumii}{10}
  \tightlist
  \item
    Leaves \textless{}3cm long, white woolly below, buds without
    scales\ldots{} \emph{Dryas} (octopetala)
  \item
    Leaves \textgreater{}5cm long, not white woolly below, buds with
    scales\ldots{} \emph{Gaultheria} (shallon)
  \end{enumerate}
\item
  Leaves gland pitted below\ldots{} \emph{Vaccinium} (vitis-idaea)
\item
  Leaves not gland pitted below -\textgreater{} 15

  \begin{enumerate}
  \def\labelenumii{\arabic{enumii}.}
  \setcounter{enumii}{14}
  \tightlist
  \item
    Leaves few and clustered at stem apices\ldots{} \emph{Gaultheria}
    (procumbens)
  \item
    Leaves many and distributed along stems\ldots{} \emph{Phyllodoce}
    (caerulea)
  \end{enumerate}
\item
  Plant with peltate scales\ldots{} \emph{Eleagnus} (macrophylla)
\item
  Without peltate scales -\textgreater{} 19

  \begin{enumerate}
  \def\labelenumii{\arabic{enumii}.}
  \setcounter{enumii}{18}
  \tightlist
  \item
    Plant with stellate hairs, buds clustered at leaf apices\ldots{}
    \emph{Quercus}
  \item
    Plant without stellate hairs -\textgreater{} 21
  \end{enumerate}
\item
  Leaves white hairy below -\textgreater{} 23
\item
  Leaves not white hairy below -\textgreater{} 25

  \begin{enumerate}
  \def\labelenumii{\arabic{enumii}.}
  \setcounter{enumii}{22}
  \tightlist
  \item
    Young twigs white felted\ldots{} \emph{Brachyglottis} (x jubar)
  \item
    Young twigs brown\ldots{} \emph{Olearia} (macrodonta)
  \end{enumerate}
\item
  Leaves at least at apex, spiny -\textgreater{} 27
\item
  Leaves not spiny -\textgreater{} 29

  \begin{enumerate}
  \def\labelenumii{\arabic{enumii}.}
  \setcounter{enumii}{26}
  \tightlist
  \item
    Twigs green, leaves \textgreater{}3cm, tree\ldots{} \emph{Ilex}
  \item
    Twigs reddish, leaves \textless{}3cm, shrub\ldots{}
    \emph{Gaultheria}
  \end{enumerate}
\item
  (Young) twigs with glandular hairs or reddish bristly hairs
  -\textgreater{} 31
\item
  (Young) twigs not glandular hairy or bristly -\textgreater{} 35

  \begin{enumerate}
  \def\labelenumii{\arabic{enumii}.}
  \setcounter{enumii}{30}
  \tightlist
  \item
    Leaves aromatic with glands below\ldots{} \emph{Escallonia}
    (macrantha)
  \item
    Leaves not aromatic, no glands below -\textgreater{} 33
  \end{enumerate}
\item
  Tall shrub to 5m, leaves cuneate at base\ldots{} \emph{Arbutus}
  (unedo)
\item
  Shrub to 1.5m, leaves rounded to cordate at base\ldots{}
  \emph{Gaultheria} (shallon)

  \begin{enumerate}
  \def\labelenumii{\arabic{enumii}.}
  \setcounter{enumii}{34}
  \tightlist
  \item
    Young twigs hairy -\textgreater{} 37
  \item
    Young twigs hairless\ldots{} \emph{Prunus}
  \end{enumerate}
\item
  Leaves \textless{}5m long\ldots{} \emph{Rhamnus} (alaternus)
\item
  Leaves \textgreater{}7cm long\ldots{} \emph{Leucothoe} (fontanesia) 39
  (2). Leaves with stellate hairs -\textgreater{} 41

  \begin{enumerate}
  \def\labelenumii{\arabic{enumii}.}
  \setcounter{enumii}{39}
  \tightlist
  \item
    Leaves with simple hairs, or hairless -\textgreater{} 43
  \end{enumerate}
\item
  Stellate hairs both sides\ldots{} \emph{Lavatera}
\item
  Stellate hairs below only\ldots{} \emph{Quercus}

  \begin{enumerate}
  \def\labelenumii{\arabic{enumii}.}
  \setcounter{enumii}{42}
  \tightlist
  \item
    Leaves \textgreater{}90cm, divided to base in 30-40 lobes, in
    terminal crown, trunk covered in petiole bases. Palm\ldots{}
    \emph{Trachycarpus} (fortunei)
  \item
    Leaves \textless{}10cm, not as above -\textgreater{} 45
  \end{enumerate}
\item
  Leaves white woolly on one or both sides -\textgreater{} 47
\item
  Leaves not woolly or velvety, leaves to 25cm, palmately lobed\ldots{}
  \emph{Fatsia} (japonica)

  \begin{enumerate}
  \def\labelenumii{\arabic{enumii}.}
  \setcounter{enumii}{46}
  \tightlist
  \item
    Leaves white woolly all over, pinnately lobed, aromatic. Plant not
    rooting at nodes\ldots{} \emph{Santolina} (chamaecyparissus)
  \item
    Leaves white woolly below only -\textgreater{} 49
  \end{enumerate}
\item
  Leaves to 12cm, ovate, with translucent dots\ldots{} \emph{Rubus}
  (tricolor)
\item
  Leaves \textless{}2cm, oblong, no translucent dots\ldots{}
  \emph{Dryas} (octopetala)
\end{enumerate}

\hypertarget{key-5.3-deciduous-plants}{%
\subsection{Key 5.3: Deciduous plants}\label{key-5.3-deciduous-plants}}

\begin{enumerate}
\def\labelenumi{\arabic{enumi}.}
\tightlist
\item
  Twigs with spines, prickles or thorns -\textgreater{} Key 5.3.1
\item
  Twigs unarmed -\textgreater{} 3

  \begin{enumerate}
  \def\labelenumii{\arabic{enumii}.}
  \setcounter{enumii}{2}
  \tightlist
  \item
    Buds with bud scales -\textgreater{} 5
  \item
    Buds without bud scales, or buds hidden/obscured -\textgreater{} Key
    5.3.2
  \end{enumerate}
\item
  Leaf scars with 3 bundle traces, or in 3 distinct groups
  -\textgreater{} 7
\item
  Leaf scars not with 3 bundle traces, or leaf scars and bundle traces
  obscured/absent -\textgreater{} 9

  \begin{enumerate}
  \def\labelenumii{\arabic{enumii}.}
  \setcounter{enumii}{6}
  \tightlist
  \item
    Terminal buds present -\textgreater{} Key 5.3.3
  \item
    Terminal buds absent -\textgreater{} Key 5.3.4
  \end{enumerate}
\item
  Terminal buds present -\textgreater{} Key 5.3.5
\item
  Terminal buds absent -\textgreater{} Key 5.3.6
\end{enumerate}

\hypertarget{key-5.3.1-twigs-with-spines-prickles-or-thorns}{%
\subsubsection{Key 5.3.1: Twigs with spines, prickles or
thorns}\label{key-5.3.1-twigs-with-spines-prickles-or-thorns}}

\begin{enumerate}
\def\labelenumi{\arabic{enumi}.}
\tightlist
\item
  Plant with prickles or spines -\textgreater{} 3
\item
  Plant with thorns -\textgreater{} 13

  \begin{enumerate}
  \def\labelenumii{\arabic{enumii}.}
  \setcounter{enumii}{2}
  \tightlist
  \item
    Plants with spines, occurring beneath each bud at each node
    -\textgreater{} 5
  \item
    Plants without spines, but with prickles on stems -\textgreater{} 9
  \end{enumerate}
\item
  1 or 3(+) spines at each node -\textgreater{} 7
\item
  2 spines at each node, flattened lengthwise and buds concealed beneath
  leaf scar\ldots{} \emph{Robinia}

  \begin{enumerate}
  \def\labelenumii{\arabic{enumii}.}
  \setcounter{enumii}{6}
  \tightlist
  \item
    Leaf scars many on short shoots beneath bud, wood bright yellow
    beneath\ldots{} \emph{Berberis}
  \item
    Leaf scars singular and distinct above spines\ldots{} \emph{Ribes}
    (uva-crispa)
  \end{enumerate}
\item
  Twigs extremely stout, leaf scars half encircling twig with
  \textasciitilde{}15 bundle traces in a line\ldots{} \emph{Aralia}
  (elata)
\item
  Not as above, bundle traces 3, if visible -\textgreater{} 11

  \begin{enumerate}
  \def\labelenumii{\arabic{enumii}.}
  \setcounter{enumii}{10}
  \tightlist
  \item
    Plant with persistant petiole base, where leaf has torn off\ldots{}
    \emph{Rubus}
  \item
    Leaf scars present, almost linear\ldots{} \emph{Rosa} 13 (2). Twigs
    with peltate scales -\textgreater{} 15
  \end{enumerate}
\item
  Twigs never with peltate scales -\textgreater{} 17

  \begin{enumerate}
  \def\labelenumii{\arabic{enumii}.}
  \setcounter{enumii}{14}
  \tightlist
  \item
    Twigs without terminal buds, grey to bronze brown scales\ldots{}
    \emph{Hippophae} (rhamnoides)
  \item
    Twigs with terminal buds (in at least some twigs), olive brown to
    silver scales\ldots{} \emph{Eleagnus} (umbellatus)
  \end{enumerate}
\item
  Twigs never with terminal buds -\textgreater{} 19
\item
  Twigs with terminal buds -\textgreater{} 21

  \begin{enumerate}
  \def\labelenumii{\arabic{enumii}.}
  \setcounter{enumii}{18}
  \tightlist
  \item
    Buds mostly lateral to thorns\ldots{} \emph{Chaenomeles}
  \item
    Buds, often many, on thorns themselves. (something about collateral
    buds?)\ldots{} \emph{Prunus} (spinosa) {[}{[}21. Thorns not very
    sharp\ldots{} \emph{Malus}
  \end{enumerate}
\item
  Thorns sharp -\textgreater{} 23

  \begin{enumerate}
  \def\labelenumii{\arabic{enumii}.}
  \setcounter{enumii}{22}
  \tightlist
  \item
    Twigs densly long, patent hairy\ldots{} \emph{Mespilus} (germanica)
  \item
    Twigs not hairy or adpressed hairy -\textgreater{} 25
  \end{enumerate}
\item
  Buds purplish, acute\ldots{} \emph{Pyrus}
\item
  Buds rounded, reddish\ldots{} \emph{Crataegus}{]}{]} IMPROVE. FIND
  BUDS.
\end{enumerate}

\hypertarget{key-5.3.2-buds-without-bud-scales-or-buds-hidden.}{%
\subsubsection{Key 5.3.2: Buds without bud scales, or buds
hidden.}\label{key-5.3.2-buds-without-bud-scales-or-buds-hidden.}}

\begin{verbatim}
    Buds without bud scales, or buds at least partly obscured
        With terminal buds
            LABURNUM
            RHAMNUS (FRANGULA)
            PARROTIA 
            PTEROCARYA
            JUGLANS

        Without terminal buds
            RHUS - leaf scar small and almost encircling bud. Buds with long hairs, blunt conical. Twigs hairy and with milky sap. Pith ochre yellow. Bundle traces 3.
            ROBINIA - leaf scar large, buds covered initially by a membrane, twigs angled. Bundle traces 3.
            BUDDLEJA (ALTERNIFOLIA) - deciduous, soft wooded shrubs, stems 4 ridged. Twigs almost hairless, but with some stellate hairs (and buds). 1 pair of exposed scales. Bundle trace 1.
            COTONEASTER - shrubs to small trees, twigs rounded. Pith continuous. Bundle trace 1, indistinct, stipules mostly persistent
            SPARTIUM - twigs green, elongated, striate ridged. Pith continuous. Buds shorter than elongated leaf cushion. Leaf scars minute and bundle trace 1.
            GENISTA? - twigs green, ribbed or grooved. Pith continuous. Buds small, solitary, sessile and ovoid ~6 scales. BT 1.
            CYTISUS - twigs green, terete, ribbed or grooved. Pith continuous. Buds small with ~4 scales, often hidden behind leaf scars. Bundle traces 1. Stipules or stipule scars at the top of the leaf cushion.
\end{verbatim}

\begin{enumerate}
\def\labelenumi{\arabic{enumi}.}
\tightlist
\item
  Terminal buds present -\textgreater{} 3
\item
  Terminal buds absent -\textgreater{} 11
\item
  Pith chambered -\textgreater{} 5
\item
  Pith continuous -\textgreater{} 7

  \begin{enumerate}
  \def\labelenumii{\arabic{enumii}.}
  \setcounter{enumii}{4}
  \tightlist
  \item
    Terminal bud leaves spreading, \textgreater{}15mm long\ldots{}
    \emph{Pterocarya} (fraxinifolia)
  \item
    Terminal bud leaves compact, \textless{}15mm long\ldots{}
    \emph{Juglans}
  \end{enumerate}
\item
  Twigs with at least some stellate hairs, buds dark brown\ldots{}
  \emph{Parrotia} (persica)
\item
  Twigs never with stellate hairs -\textgreater{} 9

  \begin{enumerate}
  \def\labelenumii{\arabic{enumii}.}
  \setcounter{enumii}{8}
  \tightlist
  \item
    Stipules persistent on large leaf cushions, buds silvery
    hairy\ldots{} \emph{Laburnum}
  \item
    Stipules minute, if present, buds densely brown
    hairy\ldots{}\emph{Rhamnus} (frangula)
  \end{enumerate}
\end{enumerate}

11 (2). Buds obscured, either by remains of leaf bases or by a membrane
-\textgreater{} 13 12. Buds entirely unobscured -\textgreater{} 19 13.
Buds obscured by a membrane, later splitting. Bundle traces 3\ldots{}
\emph{Robinia} (pseudo-acacia) 14. Buds obscured by leaf bases, bundle
trace 1 -\textgreater{} 15 15. Twigs finely striated, rush like\ldots{}
\emph{Spartium} (junceum) 16. Twigs angled winged, or grooved
-\textgreater{} 17 17. Twigs 5 angled or winged\ldots{} \emph{Cytisus}
18. Twigs furrowed with \textgreater{}5 angles, the slender twigs 4
angled\ldots{} \emph{Genista} (CHECK) 19. Stems 4 ridged, with some
stellate hairs\ldots{} \emph{Buddleja} 20. Stems not 4 ridged
-\textgreater{} 21 21. Leaf scar almost encircling the bud. Stipules
absent. Twigs very hairy with milky sap. Bundle traces 3\ldots{}
\emph{Rhus} 22. Leaf scar tiny, not encircling bud. Stipules mostly
persistent. Twigs hairy or not, no sap. Bundle trace 1\ldots{}
\emph{Cotoneaster}

\hypertarget{key-5.3.3-leaf-scars-with-3-bundle-traces-terminal-buds-present.}{%
\subsubsection{Key 5.3.3: Leaf scars with 3 bundle traces, terminal buds
present.}\label{key-5.3.3-leaf-scars-with-3-bundle-traces-terminal-buds-present.}}

(buds), (leaf scars), (twigs)

\begin{verbatim}
With three bundle traces
    Buds with bud scales
        With terminal buds
        JUGLANS - lateral buds often superposed, blunt to acute with broad. Bud scales thin and scaly, rounded at tip. Leaf scars large and as broad as twig, often ± 3 lobed, with 3 groups of semi circular traces. Twigs thick, slightly angular, hairless.
        ALNUS - lateral buds spiral in 1/3 divergence. First bud scale turned towards the stem (adaxial). Buds stalked. Number and shape of bundle traces variable and leaf scars too. Twigs and pith often 3 angular. Infructescences are persistent and cone like.
        BETULA - only short shoots with terminal buds, with 5-8 scales on lateral short shoots and 3-5 scales on lateral buds.
        STACHYURUS - buds dimorphic. inflorescence branches well developed in racemes. Terminal buds ~3mm long, laterals 2mm. 2-3 outer bud scales. Leaf scars with 3 traces, central larger, laterals less distinct. Twigs slender, glabrous, red to olive brown, long and curving down.
        PRUNUS - usually ~6 exposed scales, buds collaterally multiple, elongated ovoid or conical. Terminal buds present (cherries, almonds, peaches) or absent (apricots,plums and sloes). Bundle traces 3, minute, scars half round, elliptical. Pith pale or brown, solid. May have to key out last...
        RIBES - buds scales spiral , simple, rather loose, acute, short stalked. Bark shedding. Leaf scars u shaped or broadly crescent shaped. Twigs solid to spongey, decurrently ridged from nodes 
        ARONIA - red buds, long acute terminal bud, oblong, flattened and appressed. 5 abruptly pointed glandular denticulate scales. Leaf scars u shaped, low. Twigs with solid pith.
        FAGUS - buds long fusiform with scales in 4 rows (decussate), divergent and oblique over leaf scars, >10 scales. Leaf scars half round, small, bundle traces 3. Stipular scars almost encircling twig. Twigs, slender, zig-zag, round.
        POPULUS - buds dimorphic. Terminal buds surrounded by adaxial stipule scales in a spiral. Lateral buds the first bud scale is opposite the principal axis (i.e lowermost bud scale immediately over leaf scar), leaf scar mostly broadly crescent shaped to triangular, twigs rounded, rarely 5 angled.
        AMELANCHIER - buds elongated with half a dozen scales (sometimes more), reddish, acute. Leaf scars narrowly crescent/U shaped. Twigs slender, zig-zag.
        COTINUS - buds bluish-white with waxy bloom, two valvate bud scales. Leaf scars crescent shaped to 3 lobed. Twigs with yellow wood when scratched, with lenticels, hairless, with waxy bloom too.
        MESPILUS - Buds acute ovoid with >6 bud scales, ciliate. Leaf scars small, dark, crescent shaped and indistinctly 3 traced. Twigs towards the tip densly spreading hairy.
        PYRUS - terminal buds of long shoots are more densly hairy. Basal bud scales of terminal buds acute and keeled. Short shoots with terminal buds. Leaf scars small, blackish. Twigs mostly glabrous.
        MALUS - unsatisfactory at present...
        LABURNUM - bud scales very silver hairy, terminal 3-4mm long. Leaf scar with stipule remains at base, transversely elliptical with 3 bundle traces. Short and long shoots present. Twigs slender, slightly fluted. Pith solid, white.
        LIQUIDAMBAR - buds ovoid, acute with 5-6 bud scales, glossy, reddish with thin ciliate margins. Leaf scars half elliptical, little raised on cushions, with 3 bundle traces. Twigs sometimes with corky ridges, somewhat angled, pith solid, star shaped.
        DAVIDIA - buds acute ovoid, with round base, glossy dark wine red with up to 6 scales. Leaf scars half elliptical to 3 lobed with 3 bundle traces. Twigs hairless, thick, little zig-zag and pith solid with firmer plates at intervals.
        SORBUS - buds variable, sub conical or oblong (globose in torminalis), with terminal buds larger than laterals with inner scales with many hairs and matted with gum, or hairless, bud scales often with a brown margin. Leaf scars with 3 or 5 bundle traces. Leaf cushions different colour to twigs(?).
        The Sorbus species with 3 bundle traces are:
          Sorbus torminalis - green, glossy bud scales with brown margins, terminal bud blunt (terminal buds 6mm)
          Sorbus aria (agg) - Woolly buds, green with brown margin (terminal buds 10mm)
\end{verbatim}

\begin{enumerate}
\def\labelenumi{\arabic{enumi}.}
\tightlist
\item
  Pith chambered\ldots{} \emph{Juglans}
\item
  Pith not chambered -\textgreater{} 3

  \begin{enumerate}
  \def\labelenumii{\arabic{enumii}.}
  \setcounter{enumii}{2}
  \tightlist
  \item
    Flower buds numerous on elongated racemes (hortal)\ldots{}
    \emph{Stachyurus} (praecox)
  \item
    Flower buds if present, different -\textgreater{} 5
  \end{enumerate}
\item
  Lateral buds short to long stalked -\textgreater{} 7
\item
  Lateral buds sessile -\textgreater{} 9

  \begin{enumerate}
  \def\labelenumii{\arabic{enumii}.}
  \setcounter{enumii}{6}
  \tightlist
  \item
    Tree. Lateral buds long stalked, with 2(3) valvate scales\ldots{}
    \emph{Alnus}
  \item
    Small shrub. Lateral buds short stalked, with \textgreater{}5
    scales\ldots{} \emph{Ribes}
  \end{enumerate}
\item
  Trunk and boughs of tree with transverse stripes. Buds often
  collateral\ldots{} \emph{Prunus}
\item
  Not as above -\textgreater{} 11

  \begin{enumerate}
  \def\labelenumii{\arabic{enumii}.}
  \setcounter{enumii}{10}
  \tightlist
  \item
    Only short shoots with terminal buds (5-8 scaled), lateral buds 3-5
    scaled\ldots{} \emph{Betula}
  \item
    Either long shoots or both long and short shoots with terminal buds
    -\textgreater{} 13
  \end{enumerate}
\item
  Buds clearly dimorphic. Terminal buds: spiral bud scales, lateral
  buds: distichous scales with lowest above leaf scar\ldots{}
  \emph{Populus}
\item
  Buds similar (terminal buds may be larger), or not as above
  -\textgreater{} 15

  \begin{enumerate}
  \def\labelenumii{\arabic{enumii}.}
  \setcounter{enumii}{14}
  \tightlist
  \item
    Terminal buds elongated, slender, acute -\textgreater{} 17
  \item
    Terminal buds less elongate, more squat -\textgreater{} 21
  \end{enumerate}
\item
  Buds divergent, oblique over leaf scars. \textgreater{}10 bud scales
  exposed\ldots{} \emph{Fagus}
\item
  Bud more appressed to twig, \textless{}10 bud scales exposed
  -\textgreater{} 19

  \begin{enumerate}
  \def\labelenumii{\arabic{enumii}.}
  \setcounter{enumii}{18}
  \item
    5 or fewer glandular denticulate scales on buds\ldots{}
    \emph{Aronia}
  \item
    \begin{quote}
    5 scales on buds\ldots{} \emph{Amelanchier}
    \end{quote}
  \end{enumerate}
\item
  Bud scales 2, valvate. Twigs pruinose, wood yellow\ldots{}
  \emph{Cotinus} (coggyria)
\item
  Bud scales \textgreater{}2, twigs not pruinose, wood not yellow
  -\textgreater{} 23

  \begin{enumerate}
  \def\labelenumii{\arabic{enumii}.}
  \setcounter{enumii}{22}
  \tightlist
  \item
    Twigs hairy -\textgreater{} 25
  \item
    Twigs hairless -\textgreater{} 33
  \end{enumerate}
\item
  Stipules persistent on leaf scar cushion. Buds densly silver
  hairy\ldots{} \emph{Laburnum}
\item
  Stipules not persistent on leaf scar cushion -\textgreater{} 27

  \begin{enumerate}
  \def\labelenumii{\arabic{enumii}.}
  \setcounter{enumii}{26}
  \tightlist
  \item
    Buds with at least some white hairs -\textgreater{} 29
  \item
    Buds with hairs not white or absent -\textgreater{} 35
  \end{enumerate}
\item
  Bud scales green with brown margin, buds raised on petiole base
  differently coloured to rest of twig\ldots{} \emph{Sorbus}
\item
  Not as above -\textgreater{} 31

  \begin{enumerate}
  \def\labelenumii{\arabic{enumii}.}
  \setcounter{enumii}{30}
  \tightlist
  \item
    Lateral buds appresssed\ldots{} \emph{Malus}
  \item
    Lateral buds spreading -\textgreater{} 33
  \end{enumerate}
\item
  Leaf scars raised equal to bud width, terminal bud larger to
  12mm\ldots{} \emph{Pyrus} and \emph{Malus}
\item
  Leaf scars raised, \textgreater{} bud width, spreading hirsute to twig
  apex, terminal bud to 6mm\ldots{} \emph{Mespilus}

  \begin{enumerate}
  \def\labelenumii{\arabic{enumii}.}
  \setcounter{enumii}{34}
  \tightlist
  \item
    Buds with golden hairs at tip, otherwise brown, acute
    conical\ldots{} \emph{Pyrus}
  \item
    Buds mostly hairless, buds either glossy or sticky -\textgreater{}37
  \end{enumerate}
\item
  Buds shiny dark purple, twigs with indisinct white partitions\ldots{}
  \emph{Davidia}\\
\item
  Buds sticky, green to brown. Pith 5 angled, not partitioned\ldots{}
  \emph{Liquidambar}
\end{enumerate}

\hypertarget{key-5.3.4-leaf-scars-with-3-bundle-traces-terminal-buds-absent}{%
\subsubsection{Key 5.3.4: Leaf scars with 3 bundle traces, terminal buds
absent}\label{key-5.3.4-leaf-scars-with-3-bundle-traces-terminal-buds-absent}}

\begin{verbatim}
  TILIA - bud scales 2(3), mostly green or red glistening. Leaf scars 2 ranked on shoots, stipule scars with one of the pair much elongated, bundle traces 3 or compound and then scattererd. Twigs zig-zag, pith solid.
        BETULA - only short shoots with terminal buds, with 5-8 scales on lateral short shoots and (2)3-5 scales on lateral buds. Leaf scars 2 ranked, half elliptical, small, bundle traces 3, stipule scars narrow. Twigs, slender, zig-zag, round, pith minute, 3 sided, solid.
        CYDONIA - buds with 1-2 exposed bud scales, appressed. Leaf scars small, shallow, u shaped, bundle traces 3, stipule scars small and elongated. Twigs gray woolly, pith small, solid.
        ULMUS - buds solitary, ovoid, oblique to leaf scar, bud scales ~6, 2 ranked. Leaf scars 2 ranked, broadly crescent shaped or half round, bundle traces 3 or in 3 groups, stipule scars unequal. Twigs slender, zig-zag, rounded, pith solid.
        CORYLUS - buds with 4-6 scales, obliquely sessile. Leaf scars 2 ranked anf half round/ triangular, bundle traces 3 or multiplied and obscure. 
        CARPINUS -
        OSTRYA -
        BACCHARIS -
        MYRICA -
        COLUTEA -
        KERRIA -
        SORBARIA -
        PRUNUS -
        RHAMNUS -
        SALIX -
    
\end{verbatim}

\hypertarget{key-5.3.5}{%
\subsubsection{Key 5.3.5}\label{key-5.3.5}}

\begin{verbatim}
        GINGKO
        TAXODIUM
        LARIX
        DAPHNE
        RHODODENDRON
        DAPHNE
        ELAEGNUS
        FICUS
        LIRIODENDRON
        MAGNOLIA
        ALNUS
        SORBUS
        FAGUS
        PAEONIA
        QUERCUS
        POTENTILLA
\end{verbatim}

\hypertarget{key-5.3.6}{%
\subsubsection{Key 5.3.6}\label{key-5.3.6}}

\begin{verbatim}
        NOTHOFAGUS
        TAMARIX
        MORUS
        PLATANUS
        CASTANEA
        TILIA
        AILANTHUS
        GENISTA
        CYTISUS
        TAXODIUM
        SPIRAEA
        HIPPOCREPIS
        COTONEASTER
        POTENTILLA
\end{verbatim}

Lvs simple

\begin{verbatim}
K = Simple, entire, alt (only helpful for evergreen taxa)
N = Simple, toothed, alt (only helpful for evergreen taxa)
\end{verbatim}

TWIGS, PITH, BUDS, LEAF SCARS, BT'S

Olearia -

KA - paniculata \#EVERGREEN - leaves with peltate scales, entire,
alternate. Net veined. Unarmed NC - macrodonta \#EVERGREEN - leaves
without peltate scales, toothed, alterate. Net veined. Twigs brown,
angled. Upturned spine like teeth on leaves

Eleagnus

KA - umbellata \#EVERGREEN NB, NC - macrophylla

Hippophae

KA - rhamnoides

Quercus

KA - ilex \#EVERGREEN NC, ND, PA - suber, ilex \#EVERGREEN PA - cerris
PA - robus PA - petraea PA - rubra PA - coccinea

Buddleja

KA - alternifolia

Cotoneaster

KB - many species SOME \#EVERGREEN

Frangula

KB - alnus

Salix

KB - many KF

Photinia

KB - davidiana \#EVERGREEN

Cydonia

KB - oblonga

Fagus

KB - sylvatica

Pyrus

KB - salicifolius

Brachyglottis

KB, NC - x jubar \#EVERGREEN

Rhododendron

KB - ponticum \#EVERGREEN KB, NF - luteum

Pittosporum

KB - tobira \#EVERGREEN KB - tenuifolium \#EVERGREEN KC - crassifolium
\#EVERGREEN

Berberis

KB - many sp

Laurus

KB - nobilis \#EVERGREEN

Bupleurum

KB - fruticosum \#EVERGREEN

Vaccinium

KB, NF - corymbosum

Cotinus

KB - coggyria

Skimmia

KC - japonica \#EVERGREEN

Atriplex

KC - halimus \#EVERGREEN

Ilex

KC, ND - x altaclerensis \#EVERGREEN ND - aquifolium \#EVERGREEN

Griselinia

KC - littoralis \#EVERGREEN

Tamarix

KC - gallica KC - africana

Lycium

KC - barbarum KC - chinense

Berberis

BER - many species SOME \#EVERGREEN

Spartium

KC - junceum

Genista

KC - aetnensis KC - tenera KF - tinctoria KF - hispanica occidentalis KF
- anglica KF - pilosa

Hedera -

KD - helix etc

Muehlenbeckia

KD - complexa

Fallopia

KD - baldschuanica

Empetrum

KE - nigrum \#EVERGREEN

Ledum

KE - palustre \#EVERGREEN

Daboecia

KE - cantabrica \#EVERGREEN

Vaccinium

KE - vitis-idaea \#EVERGREEN KE - macrocarpon \#EVERGREEN KE - oxycoccos
\#EVERGREEN KE - microcarpon \#EVERGREEN KF - uliginosum

Andromeda

KE - polifolia \#EVERGREEN

Herniaria

KF - ciliolata \#EVERGREEN???

Suadea

KF - vera \#EVERGREEN

Artemisia

KF - campestris maritima \#EVERGREEN??

Sedum

KF - praealtum \#EVERGREEN

Daphne

KF - laureola \#EVERGREEN KF - mezereum

Iberis

KF - sempervirens \#EVERGREEN

Arctostaphylos

KF - uva-ursi \#EVERGREEN

Diapensia

KF - lapponica \#EVERGREEN

\begin{verbatim}
Lvs toothed
\end{verbatim}

Chaenomeles

NA - japonica NA - speciosa

Berberis

NA - many sp

Ribes

NA, PB - uva-crispa PB - rubrum PB - spicatum PB - alpinus PB - odoratum
PB - nigrum PB -sanguineum

Pyracantha

NA - coccinea \#EVERGREEN

Rhamnus

NA - cathartica ND - alaternus

Prunus

NA - spinosa ND - laurocerasus, lusitanica \#EVERGREEN NF - many sp

Crataegus

NA - mollis NA - persimilis

Malus

NA - sylvestris NF - domestica

Pyrus

NA - communis NA - pyraster NA - cordata

Dryas

NB - octopetala \#EVERGREEN

Gaultheria

NB, ND - shallon \#EVERGREEN NB - procumbens \#EVERGREEN ND - mucronata

Salix

NB - many sp NF - many sp

Arctostaphylos

NB - alpinus

Vaccinium

NB - myrtillus NB - vitis-idaea \#EVERGREEN

Phyllodoce

NB - caerulea \#EVERGREEN

Tilia

NB - petiolaris, tomentosa NE - many sp NF - many sp

Populus

NC - many sp NF - many sp PA - many sp

Sorbus

NC - many sp PA - many sp PA - torminalis

SD - aucuparia, domestica, cashmiriana, pseudohupehensis

Escallonia \#EVERGREEN

ND - macrantha

Arbutus \#EVERGREEN

ND - unedo

Leucothoe \#EVERGREEN

ND - fontanesiana

Parrotia

NE - persica

Alnus

NE - glutinosa, incana, rubra NF - cordata

Castanea

NE - sativa

Corylus

NE - avenalla, colurna PB - avenalla

Betula

NE - many sp

Ulmus

NE - many sp

Carpinus

NE - betulus

Ostrya

NE - carpinifolia

Kerria

NE - japonica

Fagus

NE - sylvatica

Nothofagus

NE - obliqua, x dodecaphleps, alpina

Amelanchier

NF - lamarckii

Myrica

NF - gale, pensylvanica

Baccharis

NF - halmifolia

Spiraea

NF - many sp PB - many sp

Morus

NF - nigra

Mespilus

NF - germanica

Aronia

NF - arbutifolia, melanocarpa

Amelanchier

NF - lamarckii

\hypertarget{section-1}{%
\subsubsection{}\label{section-1}}

leaves lobed

Lavatera

PA - arborea PB - x clementii \#EVERGREEN

Platanus

PA - x hispanica, orientalis

Trachycarpus

PA - fortunei \#EVERGREEN

Ficus

PA, PB - carica

Liriodendron

PA - tulipifera

Crataegus

PB - monogyna, laevigata

Santolina

PB - chamaecyparissus \#EVERGREEN

Rubus

PB - tricolor \#EVERGREEN PB - odoratus PB - parviflorus

RF - all other brambles SA - cockburnianus, idaeus

Dryas

PB - octopetala \#EVERGREEN

Fatsia

PB - japonica \#EVERGREEN

Vitis

PC - vinifera, coignetiae

Parthenocissus

PC - tricuspidata RH - inserta, quinquefolia

Solanum

PC - dulcamara

Laburnum

RA - anagryoides, x watereri, alpinum

Ulex

see key 2

Leaves compound

Choisya

already keyed

Medicago

RG - arborea

Akebia

RH - quinata

Rosa

SA - many sp

Robinia

SB - pseudoacacia

Juglans

SB, SD - regia

Coronilla

SB - valentina \#EVERGREEN

Potentilla

SB - fruticosa

Hippocrepis

SB - emerus

Cotulea

SB - arborescens

Rhus

SD - typina

Pterocarya

SD - fraxinifolia

Ailanthus

SD - altissima

Mahonia

SD - aquifolium, x media \#EVERGREEN

Sorbaria

SD - sorbifolia, kirilowii, tomentosa

Aralia

TA - elata

Acacia

TA - dealbata \#EVERGREEN

Lupinus

RE - arborea

Acaena?

Cut text

\begin{enumerate}
\def\labelenumi{\arabic{enumi}.}
\setcounter{enumi}{16}
\item
  Bundle traces many (9 or more) in an elliptical, arc shaped or
  U-shaped series -\textgreater{} 19
\item
  Bundle traces fewer (7 or less). Buds not superposed. -\textgreater{}
  21

  \begin{enumerate}
  \def\labelenumii{\arabic{enumii}.}
  \setcounter{enumii}{18}
  \tightlist
  \item
    Pith chambered \ldots{} \emph{Paulownia} (\textgreater{}3)
  \item
    Pith not chambered \ldots{} \emph{Fraxinus} (\textgreater{}3)
  \end{enumerate}
\item
  Buds naked often behind a persistent petiole base \ldots{}
  \emph{Cornus} (3)
\item
  Buds with 5-6 scales (thorny?) \ldots{} \emph{Rhamnus} (3)

  \begin{enumerate}
  \def\labelenumii{\arabic{enumii}.}
  \setcounter{enumii}{22}
  \tightlist
  \item
    Terminal buds very large, at least 2x as large as laterals, 20mm+.
    Leaf scars very large, distinct \ldots{} \emph{Aesculus}
    (\textgreater{}3)
  \item
    Terminal buds not much larger than laterals, or absent
    -\textgreater{} 25
  \end{enumerate}
\item
  Stipule scars large and obvious ± 1/4 to 1/2 size of leaf scars
  \ldots{} \emph{Staphylea} (3)
\item
  Stipule scars absent or not obvious -\textgreater{} 27

  \begin{enumerate}
  \def\labelenumii{\arabic{enumii}.}
  \setcounter{enumii}{26}
  \tightlist
  \item
    Buds with a pair of scales meeting in the middle or ± so - buds
    globose, red \ldots{} \emph{Viburnum} (3)
  \item
    Buds with \textgreater{} 2 scales or buds naked -\textgreater{} 29
  \end{enumerate}
\item
  Terminal buds or all buds naked -\textgreater{} 31
\item
  All buds with scales -\textgreater{} 35

  \begin{enumerate}
  \def\labelenumii{\arabic{enumii}.}
  \setcounter{enumii}{30}
  \tightlist
  \item
    Naked buds stellate scurfy \ldots{} \emph{Viburnum} (3)
  \item
    Naked buds not stellate scurfy -\textgreater{} 33
  \end{enumerate}
\item
  Bundle traces 3. Lateral buds \textgreater{}5mm, adpressed, hairy
  \ldots{} \emph{Cornus} (3)
\item
  Bundle traces \textgreater{}3. Lateral buds minute \textless{}2mm
  \ldots{} \emph{Hydrangea} (3, but check all Hydrangeas)
\end{enumerate}

Vegetative key to Quercus species:

Quercus - buds with 5-ranked scales, clustered towards end of twig.
Fruits nuts held in a cupule.

\begin{enumerate}
\def\labelenumi{\arabic{enumi}.}
\tightlist
\item
  Leaves evergreen, white to grey woolly below at least when young
  -\textgreater{} 3
\item
  Leaves deciduous -\textgreater{} 5

  \begin{enumerate}
  \def\labelenumii{\arabic{enumii}.}
  \setcounter{enumii}{2}
  \tightlist
  \item
    Bark corky, leaves with 5-8 spine tipped teeth \ldots{} Quercus
    suber
  \item
    Bark not corky, leaves only spined on young shoots \ldots{} Quercus
    ilex
  \end{enumerate}
\item
  Terminal bud cluster with long, narrow, persistent stipules \ldots{}
  Quercus cerris
\item
  Terminal bud cluster with without long, narrow, persistent stipules
  -\textgreater{} 7

  \begin{enumerate}
  \def\labelenumii{\arabic{enumii}.}
  \setcounter{enumii}{6}
  \tightlist
  \item
    Leaves lobed with aristate teeth. Hairless except for tufts of
    orange stellate hairs in vein axils below. Petiole
    \textgreater{}2.5cm -\textgreater{} 9
  \item
    Leaves not toothed with aristate teeth. hairs present or not, but
    not as above. Petiole \textless{}2.5cm -\textgreater{} 11
  \end{enumerate}
\item
  Leaves dull above, lobes \textless{} half way to midrib. Hair tufts
  inconspicuous \ldots{} Quercus rubra
\item
  Leaves shiny above, lobes \textgreater{} half way to midrib. Hair
  tufts obvious \ldots{} Quercus coccinea

  \begin{enumerate}
  \def\labelenumii{\arabic{enumii}.}
  \setcounter{enumii}{10}
  \tightlist
  \item
    Petiole to 1cm. Leaves cordate with reflexed auricles at base. Dull
    dark green above. Hairless or simple hairs below \ldots{} Quercus
    robur
  \item
    Petiole \textgreater{}1cm. Leaves cuneate, auricles absent. Shiny
    dark green above. Hairs stellate below \ldots{} Quercus petraea
  \end{enumerate}
\end{enumerate}

Glossary: Stipules - small herbaceous organs at base of buds or petiole
Stellate - star shaped Aristate - with an awn or bristle Cordate - heart
shaped, with base rounded Auricules - small lobed appendage Cuneate -
tapered to a base

\bibliography{EndnoteLib2.bib}


\end{document}
